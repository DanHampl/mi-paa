
\NeedsTeXFormat{LaTeX2e}
\LoadClass{article}
\RequirePackage{todonotes}
\RequirePackage[parfill]{parskip}
\RequirePackage[margin=2.8cm]{geometry}
\RequirePackage{hyperref}
\RequirePackage[english]{babel}
\RequirePackage{pgfplots}
\RequirePackage{listings}
\RequirePackage{amsfonts}
\RequirePackage{subfiles}
\RequirePackage{mathtools}
\RequirePackage[noabbrev,capitalize,nameinlink]{cleveref}

\providecommand{\tightlist}{%
  \setlength{\itemsep}{0pt}\setlength{\parskip}{0pt}}

\begin{document}
\title{\textbf{MI-PAA~--~Task 4}\\
Genetic algorithm solving the optimisation version of the 0/1 knapsack problem}
\author{Daniel Hampl (hampldan)}
\date{\today}
\maketitle

\tableofcontents
\newpage

\section{Introduction}
The knapsack problem is one of the most widespread NP-Complete problems. It can be described as having a knapsack with a limited capacity and multiple items, where each of these items has a set value and weight. Our task is to fill the knapsack with items of highest combined value possible.

In this paper, we will be focussing approximation algorithms solving the knapsack problem and the deviation of results provided by these approximation algorithms.

\subsection{Definition\cite{WEBSITE:knapsackDef}}
We have weight $W$ and $n$ items, where item $i$ is described as a pair $(w_i, v_i)$.

\begin{itemize}
    \item $w_i$ is a weight of object $n_i$
    \item $v_i$ is a value of object $n_i$
    \item $w_i, v_i \in \mathbb{N}\setminus\{0\}$
\end{itemize}

Find value $V$, where:

\begin{itemize}
    \item $\sum_i(x_i*v_i) = V$
    \item $\sum_i(x_i*w_i) \leq W$
    \item $x_i \in \{0,1\} \forall i$
\end{itemize}

\subsection{Task}

Choose one heuristics (simulated annealing, genetic algorithm or tabu search)
and use it for solving the optimisation version of the 0/1 knapsack problem.
Test the solution on instances with at least 30 elements. Furthermore, test
your algorithm with different configuration variables.



\section{Implementation}
All of these algorithms are implemented in Python 3.7, and all data were gathered on the Windows10
OS running on Intel(R) Core(TM) i7-6700HQ CPU @ 2.60GHz.

In our genetic algorithm, we have used a fixed pool size for each to be 30, and the number of
generations has been set to ten times the size of the concrete instance.

\subsection{Fitness function}
For calculating the fitness of each configuration, we have used the price of this combination
with two additional conditions. The first condition being a case where the weight of said
combination is zero, and the second is the weight being higher, than the capacity of our knapsack.
In these cases, the value of our fitness function is set to 0 and -1, respectively.

\subsection{Selection of the next generation}
For selecting the next generation, we have decided to keep 2 elements from the last generation
as they are. As for the rest of the new generation, we have selected 10\% of the previous
generation and combined them with the last generation using random breeding.

The random breeding takes 2 combinations and randomly selects each bit from one of those,
which in turn creates a new combination used further in our algorithm.

\subsection{Mutation}
In every breeding, there is a chance of a mutation. During the selection of bits from parents,
there is a chance to flip the bit. This chance is 1 in 128; however, it significantly grows with
each cycle, where our fittest element stays unchanged.

Moreover, to ensure the potency of our population, we enforce uniqueness in our pool. To not
get stuck in an infinite loop, we also increase the chance of mutation with every step,
where we attempted to add an element, which was already in our pool.



\addplot[scatter,scatter src=explicit symbolic]table[meta=label] {
x y label
4 .860686 hungryStupidDeviation
10 .566400 hungryStupidDeviation
15 .286770 hungryStupidDeviation
20 .276166 hungryStupidDeviation
};
\addplot[scatter,scatter src=explicit symbolic]table[meta=label] {
x y label
4 1.493548 singleDeviation
10 .566400 singleDeviation
15 .286770 singleDeviation
20 .276166 singleDeviation
};
\addplot[scatter,scatter src=explicit symbolic]table[meta=label] {
x y label
4 5.959800 fptasDeviation
10 3.533680 fptasDeviation
15 2.698540 fptasDeviation
20 1.992056 fptasDeviation
};


\addplot[scatter,scatter src=explicit symbolic]table[meta=label] {
x y label
4 .860686 hungryStupidDeviation
10 .566400 hungryStupidDeviation
15 .286770 hungryStupidDeviation
20 .276166 hungryStupidDeviation
};
\addplot[scatter,scatter src=explicit symbolic]table[meta=label] {
x y label
4 1.493548 singleDeviation
10 .566400 singleDeviation
15 .286770 singleDeviation
20 .276166 singleDeviation
};
\addplot[scatter,scatter src=explicit symbolic]table[meta=label] {
x y label
4 5.959800 fptasDeviation
10 3.533680 fptasDeviation
15 2.698540 fptasDeviation
20 1.992056 fptasDeviation
};




\addplot+[scatter,only marks,scatter src=explicit symbolic]table[meta=label] {
x y label
0 275328 max
1 300219 max
2 304174 max
3 304174 max
4 310966 max
5 341826 max
6 341826 max
7 341826 max
8 341826 max
9 341826 max
10 346476 max
11 370354 max
12 370354 max
13 370354 max
14 370354 max
15 370354 max
16 370354 max
17 370354 max
18 370354 max
19 370354 max
20 370354 max
21 370354 max
22 370354 max
23 370354 max
24 370354 max
25 370354 max
26 370354 max
27 370354 max
28 370354 max
29 370354 max
30 370354 max
31 370354 max
32 370354 max
33 370354 max
34 370354 max
35 370354 max
36 370354 max
37 370354 max
38 370354 max
39 373938 max
40 373938 max
41 373938 max
42 373938 max
43 377330 max
44 377330 max
45 377330 max
46 377330 max
47 377330 max
48 377330 max
49 377330 max
50 377330 max
51 377330 max
52 377330 max
53 377330 max
54 377330 max
55 377330 max
56 377330 max
57 377330 max
58 377330 max
59 377330 max
60 377330 max
61 377330 max
62 377330 max
63 377330 max
64 377330 max
65 377330 max
66 377330 max
67 377330 max
68 377330 max
69 377330 max
70 377330 max
71 377330 max
72 377330 max
73 377330 max
74 377330 max
75 377330 max
76 377330 max
77 377330 max
78 377330 max
79 377330 max
80 377330 max
81 377330 max
82 377330 max
83 377330 max
84 377330 max
85 377330 max
86 377330 max
87 377330 max
88 377330 max
89 377330 max
90 377330 max
91 377330 max
92 377330 max
93 377330 max
94 377330 max
95 377330 max
96 377330 max
97 377330 max
98 377330 max
99 377330 max
100 377330 max
101 377330 max
102 377330 max
103 380750 max
104 380750 max
105 380750 max
106 380750 max
107 380750 max
108 380750 max
109 380750 max
110 380750 max
111 380750 max
112 380750 max
113 380750 max
114 380750 max
115 380750 max
116 380750 max
117 380750 max
118 380750 max
119 380750 max
120 380750 max
121 380750 max
122 380750 max
123 380750 max
124 380750 max
125 380750 max
126 380750 max
127 380750 max
128 380750 max
129 380750 max
130 380750 max
131 380750 max
132 380750 max
133 380750 max
134 380750 max
135 380750 max
136 382189 max
137 382189 max
138 382189 max
139 382189 max
140 382189 max
141 382189 max
142 382189 max
143 382189 max
144 382189 max
145 382189 max
146 382189 max
147 382189 max
148 382189 max
149 382189 max
150 382189 max
151 382189 max
152 382189 max
153 382189 max
154 382189 max
155 382189 max
156 382189 max
157 382189 max
158 382189 max
159 382189 max
160 382189 max
161 382189 max
162 382189 max
163 382189 max
164 382189 max
165 382189 max
166 382189 max
167 382189 max
168 382189 max
169 382189 max
170 382189 max
171 382189 max
172 382189 max
173 382189 max
174 382189 max
175 382189 max
176 382189 max
177 382189 max
178 382189 max
179 382189 max
180 382189 max
181 382189 max
182 382189 max
183 382189 max
184 382189 max
185 382189 max
186 382189 max
187 382189 max
188 382189 max
189 382189 max
190 382189 max
191 382189 max
192 382189 max
193 382189 max
194 382189 max
195 382189 max
196 382189 max
197 382189 max
198 382189 max
199 382189 max
200 382189 max
201 382189 max
202 382189 max
203 382189 max
204 382189 max
205 382189 max
206 382189 max
207 382189 max
208 382189 max
209 382189 max
210 382189 max
211 382189 max
212 382189 max
213 382189 max
214 382189 max
215 382189 max
216 382189 max
217 382189 max
218 382189 max
219 382189 max
220 382189 max
221 382189 max
222 382189 max
223 382189 max
224 382189 max
225 382189 max
226 382189 max
227 382189 max
228 382189 max
229 382189 max
230 382189 max
231 382189 max
232 382189 max
233 382189 max
234 382189 max
235 382189 max
236 382189 max
237 382189 max
238 382189 max
239 382189 max
240 382189 max
241 382189 max
242 382189 max
243 382189 max
244 382189 max
245 382189 max
246 382189 max
247 382189 max
248 382189 max
249 382189 max
250 382189 max
251 382189 max
252 382189 max
253 382189 max
254 382189 max
255 382189 max
256 382189 max
257 382189 max
258 382189 max
259 382189 max
260 382189 max
261 382189 max
262 382189 max
263 382189 max
264 382189 max
265 382189 max
266 382189 max
267 382189 max
268 382189 max
269 382189 max
270 382189 max
271 382189 max
272 382189 max
273 382189 max
274 382189 max
275 382189 max
276 382189 max
277 382189 max
278 382189 max
279 382189 max
280 382189 max
281 382189 max
282 382189 max
283 382189 max
284 382189 max
285 382189 max
286 382189 max
287 382189 max
288 382189 max
289 382189 max
290 382189 max
291 382189 max
292 382189 max
293 382189 max
294 382189 max
295 382189 max
296 382189 max
297 382189 max
298 382189 max
299 382189 max
};
\addplot+[scatter,only marks,scatter src=explicit symbolic]table[meta=label] {
x y label
0 0 med
1 201745 med
2 235769 med
3 188449 med
4 181251 med
5 226842 med
6 0 med
7 215283 med
8 212220 med
9 276808 med
10 257817 med
11 280463 med
12 249289 med
13 0 med
14 0 med
15 227820 med
16 0 med
17 0 med
18 267649 med
19 279071 med
20 253179 med
21 298103 med
22 269965 med
23 288903 med
24 274424 med
25 0 med
26 298191 med
27 249244 med
28 253761 med
29 243219 med
30 240385 med
31 282240 med
32 272207 med
33 281750 med
34 283760 med
35 246136 med
36 299707 med
37 272751 med
38 0 med
39 298006 med
40 284321 med
41 274301 med
42 272113 med
43 274839 med
44 275802 med
45 285308 med
46 263262 med
47 262968 med
48 271923 med
49 259238 med
50 265803 med
51 279815 med
52 0 med
53 275542 med
54 273303 med
55 233514 med
56 277246 med
57 0 med
58 313962 med
59 276109 med
60 259028 med
61 261861 med
62 275343 med
63 276751 med
64 275817 med
65 261566 med
66 300085 med
67 275711 med
68 273004 med
69 252903 med
70 262097 med
71 279576 med
72 286948 med
73 283528 med
74 292443 med
75 274739 med
76 300741 med
77 303986 med
78 267823 med
79 0 med
80 0 med
81 0 med
82 0 med
83 278809 med
84 294248 med
85 304201 med
86 280286 med
87 271430 med
88 271138 med
89 228863 med
90 0 med
91 201088 med
92 268852 med
93 262909 med
94 274429 med
95 255351 med
96 270748 med
97 0 med
98 280248 med
99 237319 med
100 272861 med
101 291826 med
102 257436 med
103 287433 med
104 270155 med
105 0 med
106 268199 med
107 0 med
108 216035 med
109 274215 med
110 255298 med
111 286668 med
112 273397 med
113 253153 med
114 288263 med
115 264612 med
116 0 med
117 0 med
118 298007 med
119 278643 med
120 283572 med
121 284233 med
122 242934 med
123 248542 med
124 266328 med
125 253913 med
126 289542 med
127 294265 med
128 217482 med
129 269351 med
130 276044 med
131 291978 med
132 251565 med
133 263013 med
134 0 med
135 206215 med
136 0 med
137 273238 med
138 287761 med
139 227785 med
140 260742 med
141 249005 med
142 238832 med
143 231030 med
144 286277 med
145 282106 med
146 267610 med
147 312615 med
148 304691 med
149 284450 med
150 296776 med
151 267799 med
152 0 med
153 246390 med
154 287499 med
155 270593 med
156 0 med
157 261403 med
158 0 med
159 0 med
160 0 med
161 298668 med
162 293868 med
163 295823 med
164 273624 med
165 267683 med
166 248892 med
167 299366 med
168 271253 med
169 257821 med
170 261331 med
171 270831 med
172 281179 med
173 263848 med
174 284373 med
175 279968 med
176 254979 med
177 306334 med
178 0 med
179 294661 med
180 0 med
181 0 med
182 269280 med
183 278277 med
184 192198 med
185 301928 med
186 303746 med
187 248561 med
188 295980 med
189 279218 med
190 276744 med
191 283285 med
192 298752 med
193 264125 med
194 0 med
195 204354 med
196 228660 med
197 282759 med
198 292938 med
199 0 med
200 0 med
201 254004 med
202 294616 med
203 267763 med
204 261939 med
205 280984 med
206 275993 med
207 162668 med
208 263060 med
209 216125 med
210 0 med
211 0 med
212 0 med
213 201835 med
214 243650 med
215 259188 med
216 0 med
217 0 med
218 275891 med
219 268445 med
220 275442 med
221 275544 med
222 279225 med
223 258915 med
224 282673 med
225 0 med
226 305337 med
227 301614 med
228 275842 med
229 198092 med
230 289139 med
231 289236 med
232 255871 med
233 274740 med
234 293055 med
235 243948 med
236 0 med
237 220210 med
238 172574 med
239 268604 med
240 291361 med
241 0 med
242 0 med
243 0 med
244 0 med
245 251565 med
246 258377 med
247 273138 med
248 267329 med
249 247249 med
250 0 med
251 261778 med
252 256231 med
253 247918 med
254 0 med
255 246569 med
256 265977 med
257 267629 med
258 254603 med
259 268597 med
260 224847 med
261 300670 med
262 254126 med
263 210541 med
264 263191 med
265 0 med
266 255648 med
267 245263 med
268 260951 med
269 292030 med
270 272625 med
271 247939 med
272 225536 med
273 0 med
274 277011 med
275 286520 med
276 269288 med
277 0 med
278 286859 med
279 276006 med
280 0 med
281 281602 med
282 256697 med
283 290669 med
284 0 med
285 279556 med
286 0 med
287 256670 med
288 263029 med
289 242893 med
290 284463 med
291 292035 med
292 264764 med
293 294571 med
294 291453 med
295 281972 med
296 293788 med
297 277848 med
298 220317 med
299 244143 med
};
\addplot+[scatter,only marks,scatter src=explicit symbolic]table[meta=label] {
x y label
0 0 min
1 0 min
2 0 min
3 0 min
4 0 min
5 0 min
6 0 min
7 0 min
8 0 min
9 0 min
10 0 min
11 0 min
12 0 min
13 0 min
14 0 min
15 0 min
16 0 min
17 0 min
18 0 min
19 0 min
20 0 min
21 0 min
22 0 min
23 0 min
24 0 min
25 0 min
26 0 min
27 0 min
28 0 min
29 0 min
30 0 min
31 0 min
32 0 min
33 0 min
34 0 min
35 0 min
36 0 min
37 0 min
38 0 min
39 0 min
40 0 min
41 0 min
42 0 min
43 0 min
44 0 min
45 0 min
46 0 min
47 0 min
48 0 min
49 0 min
50 0 min
51 0 min
52 0 min
53 0 min
54 0 min
55 0 min
56 0 min
57 0 min
58 0 min
59 0 min
60 0 min
61 0 min
62 0 min
63 0 min
64 0 min
65 0 min
66 0 min
67 0 min
68 0 min
69 0 min
70 0 min
71 0 min
72 0 min
73 0 min
74 0 min
75 0 min
76 0 min
77 0 min
78 0 min
79 0 min
80 0 min
81 0 min
82 0 min
83 0 min
84 0 min
85 0 min
86 0 min
87 0 min
88 0 min
89 0 min
90 0 min
91 0 min
92 0 min
93 0 min
94 0 min
95 0 min
96 0 min
97 0 min
98 0 min
99 0 min
100 0 min
101 0 min
102 0 min
103 0 min
104 0 min
105 0 min
106 0 min
107 0 min
108 0 min
109 0 min
110 0 min
111 0 min
112 0 min
113 0 min
114 0 min
115 0 min
116 0 min
117 0 min
118 0 min
119 0 min
120 0 min
121 0 min
122 0 min
123 0 min
124 0 min
125 0 min
126 0 min
127 0 min
128 0 min
129 0 min
130 0 min
131 0 min
132 0 min
133 0 min
134 0 min
135 0 min
136 0 min
137 0 min
138 0 min
139 0 min
140 0 min
141 0 min
142 0 min
143 0 min
144 0 min
145 0 min
146 0 min
147 0 min
148 0 min
149 0 min
150 0 min
151 0 min
152 0 min
153 0 min
154 0 min
155 0 min
156 0 min
157 0 min
158 0 min
159 0 min
160 0 min
161 0 min
162 0 min
163 0 min
164 0 min
165 0 min
166 0 min
167 0 min
168 0 min
169 0 min
170 0 min
171 0 min
172 0 min
173 0 min
174 0 min
175 0 min
176 0 min
177 0 min
178 0 min
179 0 min
180 0 min
181 0 min
182 0 min
183 0 min
184 0 min
185 0 min
186 0 min
187 0 min
188 0 min
189 0 min
190 0 min
191 0 min
192 0 min
193 0 min
194 0 min
195 0 min
196 0 min
197 0 min
198 0 min
199 0 min
200 0 min
201 0 min
202 0 min
203 0 min
204 0 min
205 0 min
206 0 min
207 0 min
208 0 min
209 0 min
210 0 min
211 0 min
212 0 min
213 0 min
214 0 min
215 0 min
216 0 min
217 0 min
218 0 min
219 0 min
220 0 min
221 0 min
222 0 min
223 0 min
224 0 min
225 0 min
226 0 min
227 0 min
228 0 min
229 0 min
230 0 min
231 0 min
232 0 min
233 0 min
234 0 min
235 0 min
236 0 min
237 0 min
238 0 min
239 0 min
240 0 min
241 0 min
242 0 min
243 0 min
244 0 min
245 0 min
246 0 min
247 0 min
248 0 min
249 0 min
250 0 min
251 0 min
252 0 min
253 0 min
254 0 min
255 0 min
256 0 min
257 0 min
258 0 min
259 0 min
260 0 min
261 0 min
262 0 min
263 0 min
264 0 min
265 0 min
266 0 min
267 0 min
268 0 min
269 0 min
270 0 min
271 0 min
272 0 min
273 0 min
274 0 min
275 0 min
276 0 min
277 0 min
278 0 min
279 0 min
280 0 min
281 0 min
282 0 min
283 0 min
284 0 min
285 0 min
286 0 min
287 0 min
288 0 min
289 0 min
290 0 min
291 0 min
292 0 min
293 0 min
294 0 min
295 0 min
296 0 min
297 0 min
298 0 min
299 0 min
};


\addplot+[scatter,only marks,scatter src=explicit symbolic]table[meta=label] {
x y label
0 354680 max
1 354680 max
2 354680 max
3 354680 max
4 354680 max
5 390664 max
6 390664 max
7 396776 max
8 396776 max
9 396776 max
10 396776 max
11 396776 max
12 403788 max
13 403788 max
14 403788 max
15 403788 max
16 412789 max
17 412789 max
18 412789 max
19 412789 max
20 412789 max
21 412789 max
22 412789 max
23 412789 max
24 412789 max
25 433180 max
26 433180 max
27 433180 max
28 433180 max
29 433180 max
30 433180 max
31 433180 max
32 433180 max
33 433180 max
34 433180 max
35 433180 max
36 433180 max
37 433180 max
38 433180 max
39 433180 max
40 435280 max
41 435280 max
42 435280 max
43 435280 max
44 435280 max
45 435280 max
46 435280 max
47 435280 max
48 435280 max
49 435280 max
50 435280 max
51 435280 max
52 435280 max
53 435280 max
54 435280 max
55 435280 max
56 435280 max
57 435280 max
58 435280 max
59 435280 max
60 435280 max
61 435280 max
62 435280 max
63 435280 max
64 443364 max
65 443364 max
66 443364 max
67 443364 max
68 443364 max
69 443364 max
70 443364 max
71 443364 max
72 443364 max
73 443364 max
74 443364 max
75 443364 max
76 443364 max
77 443364 max
78 443364 max
79 443364 max
80 443364 max
81 447418 max
82 447418 max
83 447418 max
84 447418 max
85 447418 max
86 447418 max
87 447418 max
88 447418 max
89 447418 max
90 447418 max
91 447418 max
92 447418 max
93 447418 max
94 447418 max
95 447418 max
96 447418 max
97 447418 max
98 447418 max
99 447418 max
100 447418 max
101 447418 max
102 447418 max
103 447418 max
104 447418 max
105 447418 max
106 447418 max
107 447418 max
108 447418 max
109 447418 max
110 447418 max
111 447418 max
112 447418 max
113 447418 max
114 447418 max
115 447418 max
116 447418 max
117 447418 max
118 447418 max
119 447418 max
120 447418 max
121 447418 max
122 447418 max
123 447418 max
124 447418 max
125 447418 max
126 447418 max
127 447418 max
128 447418 max
129 447418 max
130 447418 max
131 447418 max
132 447418 max
133 447418 max
134 447418 max
135 447418 max
136 447418 max
137 447418 max
138 447418 max
139 447418 max
140 447418 max
141 447418 max
142 447418 max
143 447418 max
144 447418 max
145 447418 max
146 447418 max
147 447418 max
148 447418 max
149 447418 max
150 447418 max
151 447418 max
152 447810 max
153 447810 max
154 447810 max
155 447810 max
156 447810 max
157 447810 max
158 447810 max
159 447810 max
160 447810 max
161 447810 max
162 447810 max
163 447810 max
164 447810 max
165 448617 max
166 448617 max
167 448617 max
168 448617 max
169 448617 max
170 448617 max
171 448617 max
172 448617 max
173 448617 max
174 448617 max
175 448617 max
176 448617 max
177 448617 max
178 448617 max
179 448617 max
180 448617 max
181 448617 max
182 448617 max
183 448617 max
184 448617 max
185 448617 max
186 448617 max
187 448617 max
188 448617 max
189 448617 max
190 448617 max
191 448617 max
192 448617 max
193 448617 max
194 448617 max
195 448617 max
196 448617 max
197 449395 max
198 449395 max
199 449395 max
200 449395 max
201 449395 max
202 449395 max
203 449395 max
204 449395 max
205 449395 max
206 449395 max
207 449395 max
208 449395 max
209 449395 max
210 449395 max
211 449395 max
212 449395 max
213 449395 max
214 449442 max
215 449442 max
216 449442 max
217 450547 max
218 450547 max
219 450547 max
220 450547 max
221 450547 max
222 450547 max
223 450547 max
224 450547 max
225 450594 max
226 450594 max
227 450594 max
228 450594 max
229 450594 max
230 450594 max
231 450594 max
232 450594 max
233 450594 max
234 450594 max
235 450594 max
236 450594 max
237 450594 max
238 450594 max
239 450594 max
240 450594 max
241 450594 max
242 450594 max
243 450594 max
244 450594 max
245 450594 max
246 450594 max
247 450594 max
248 450594 max
249 455224 max
250 455224 max
251 455224 max
252 455224 max
253 455224 max
254 455224 max
255 455224 max
256 455224 max
257 455224 max
258 455224 max
259 455224 max
260 455224 max
261 455224 max
262 455224 max
263 455224 max
264 455224 max
265 455224 max
266 455224 max
267 455224 max
268 455224 max
269 455224 max
270 455224 max
271 455224 max
272 455224 max
273 455224 max
274 455224 max
275 455224 max
276 455224 max
277 455224 max
278 455224 max
279 455224 max
280 455224 max
281 455224 max
282 455224 max
283 455224 max
284 455224 max
285 455224 max
286 455224 max
287 455224 max
288 455224 max
289 455224 max
290 455224 max
291 455224 max
292 455224 max
293 455224 max
294 455224 max
295 455224 max
296 455224 max
297 455224 max
298 455224 max
299 455224 max
300 455224 max
301 455224 max
302 455224 max
303 455224 max
304 455224 max
305 455224 max
306 455224 max
307 455224 max
308 455224 max
309 455224 max
310 455224 max
311 455224 max
312 455224 max
313 455224 max
314 455224 max
315 455224 max
316 455224 max
317 455224 max
318 455224 max
319 455224 max
320 455224 max
321 455224 max
322 455224 max
323 455224 max
324 455224 max
325 455224 max
326 455224 max
327 455224 max
328 455224 max
329 455224 max
330 455224 max
331 455224 max
332 455224 max
333 455224 max
334 455224 max
335 455224 max
336 455224 max
337 455224 max
338 455224 max
339 455224 max
340 455224 max
341 455224 max
342 455224 max
343 455224 max
344 455224 max
345 455224 max
346 455224 max
347 455224 max
348 455224 max
349 455224 max
};
\addplot+[scatter,only marks,scatter src=explicit symbolic]table[meta=label] {
x y label
0 0 med
1 211699 med
2 257852 med
3 288415 med
4 285891 med
5 285040 med
6 307498 med
7 303793 med
8 302051 med
9 336141 med
10 265379 med
11 288994 med
12 293017 med
13 270079 med
14 318206 med
15 314122 med
16 302052 med
17 289715 med
18 0 med
19 0 med
20 328222 med
21 282179 med
22 256481 med
23 323417 med
24 0 med
25 254478 med
26 328543 med
27 327296 med
28 326165 med
29 329800 med
30 291460 med
31 284028 med
32 318130 med
33 322482 med
34 266980 med
35 309091 med
36 305234 med
37 312794 med
38 265272 med
39 316200 med
40 246915 med
41 259286 med
42 360626 med
43 354607 med
44 308660 med
45 329607 med
46 326342 med
47 317386 med
48 0 med
49 313123 med
50 268850 med
51 319782 med
52 336937 med
53 352685 med
54 329761 med
55 312868 med
56 311359 med
57 324762 med
58 0 med
59 260191 med
60 322595 med
61 317170 med
62 335027 med
63 0 med
64 323896 med
65 292589 med
66 321958 med
67 336464 med
68 303504 med
69 325658 med
70 332248 med
71 349668 med
72 335874 med
73 324335 med
74 351481 med
75 300578 med
76 0 med
77 0 med
78 289155 med
79 322986 med
80 0 med
81 0 med
82 311244 med
83 272157 med
84 272882 med
85 311652 med
86 324178 med
87 351385 med
88 332683 med
89 309591 med
90 330401 med
91 273772 med
92 313019 med
93 353679 med
94 331329 med
95 342899 med
96 371871 med
97 344792 med
98 380465 med
99 286071 med
100 354641 med
101 319144 med
102 0 med
103 362110 med
104 315921 med
105 283051 med
106 269878 med
107 332750 med
108 350724 med
109 309680 med
110 333235 med
111 328404 med
112 318873 med
113 326750 med
114 335405 med
115 296498 med
116 359041 med
117 322553 med
118 335187 med
119 318712 med
120 280281 med
121 0 med
122 0 med
123 314834 med
124 322474 med
125 345724 med
126 327763 med
127 331264 med
128 346662 med
129 358321 med
130 350233 med
131 325400 med
132 352242 med
133 0 med
134 315006 med
135 317810 med
136 250742 med
137 273011 med
138 285470 med
139 318549 med
140 336896 med
141 347617 med
142 348453 med
143 285091 med
144 346802 med
145 336207 med
146 330375 med
147 341497 med
148 0 med
149 301333 med
150 327356 med
151 318154 med
152 290258 med
153 329508 med
154 298411 med
155 320923 med
156 321093 med
157 283076 med
158 314129 med
159 324667 med
160 287125 med
161 354489 med
162 338838 med
163 273634 med
164 339667 med
165 309217 med
166 354392 med
167 322276 med
168 343332 med
169 259276 med
170 307100 med
171 367416 med
172 323161 med
173 343336 med
174 305503 med
175 283488 med
176 306929 med
177 343799 med
178 327021 med
179 314049 med
180 322325 med
181 306859 med
182 313699 med
183 324145 med
184 314121 med
185 325329 med
186 333240 med
187 294788 med
188 0 med
189 0 med
190 295847 med
191 330732 med
192 0 med
193 253140 med
194 331421 med
195 0 med
196 352635 med
197 329153 med
198 0 med
199 0 med
200 347300 med
201 337142 med
202 351634 med
203 341897 med
204 0 med
205 328496 med
206 351957 med
207 313260 med
208 336930 med
209 332540 med
210 337909 med
211 287113 med
212 280179 med
213 323917 med
214 357643 med
215 320949 med
216 355526 med
217 325258 med
218 315844 med
219 294106 med
220 322413 med
221 317030 med
222 350275 med
223 353321 med
224 362071 med
225 327594 med
226 300911 med
227 332889 med
228 296632 med
229 0 med
230 0 med
231 322874 med
232 296086 med
233 263362 med
234 333811 med
235 326915 med
236 304221 med
237 264371 med
238 366900 med
239 316148 med
240 265939 med
241 344436 med
242 308785 med
243 0 med
244 0 med
245 313303 med
246 274649 med
247 333320 med
248 343317 med
249 301070 med
250 237349 med
251 306156 med
252 263579 med
253 308620 med
254 0 med
255 344183 med
256 346115 med
257 325982 med
258 313155 med
259 319626 med
260 0 med
261 355008 med
262 0 med
263 291991 med
264 326758 med
265 317182 med
266 273238 med
267 359896 med
268 346850 med
269 365035 med
270 319860 med
271 328719 med
272 337036 med
273 313107 med
274 301233 med
275 350291 med
276 341462 med
277 322418 med
278 336469 med
279 0 med
280 0 med
281 307469 med
282 363612 med
283 293661 med
284 284985 med
285 349338 med
286 353902 med
287 288116 med
288 316666 med
289 323950 med
290 296233 med
291 0 med
292 313726 med
293 321381 med
294 332609 med
295 255213 med
296 285968 med
297 220965 med
298 304162 med
299 313977 med
300 304184 med
301 315584 med
302 0 med
303 330535 med
304 337051 med
305 343454 med
306 330004 med
307 307142 med
308 237350 med
309 288841 med
310 347588 med
311 301712 med
312 329757 med
313 0 med
314 0 med
315 322675 med
316 363359 med
317 321314 med
318 352183 med
319 283717 med
320 0 med
321 0 med
322 0 med
323 283704 med
324 338074 med
325 321515 med
326 0 med
327 0 med
328 306148 med
329 0 med
330 328453 med
331 312668 med
332 308933 med
333 300194 med
334 326709 med
335 317408 med
336 351398 med
337 335545 med
338 298950 med
339 328767 med
340 343406 med
341 303460 med
342 347043 med
343 268501 med
344 268011 med
345 284296 med
346 315097 med
347 352591 med
348 331572 med
349 360171 med
};
\addplot+[scatter,only marks,scatter src=explicit symbolic]table[meta=label] {
x y label
0 0 min
1 0 min
2 0 min
3 0 min
4 0 min
5 0 min
6 0 min
7 0 min
8 0 min
9 0 min
10 0 min
11 0 min
12 0 min
13 0 min
14 0 min
15 0 min
16 0 min
17 0 min
18 0 min
19 0 min
20 0 min
21 0 min
22 0 min
23 0 min
24 0 min
25 0 min
26 0 min
27 0 min
28 0 min
29 0 min
30 0 min
31 0 min
32 0 min
33 0 min
34 0 min
35 0 min
36 0 min
37 0 min
38 0 min
39 0 min
40 0 min
41 0 min
42 0 min
43 0 min
44 0 min
45 0 min
46 0 min
47 0 min
48 0 min
49 0 min
50 0 min
51 0 min
52 0 min
53 0 min
54 0 min
55 0 min
56 0 min
57 0 min
58 0 min
59 0 min
60 0 min
61 0 min
62 0 min
63 0 min
64 0 min
65 0 min
66 0 min
67 0 min
68 0 min
69 0 min
70 0 min
71 0 min
72 0 min
73 0 min
74 0 min
75 0 min
76 0 min
77 0 min
78 0 min
79 0 min
80 0 min
81 0 min
82 0 min
83 0 min
84 0 min
85 0 min
86 0 min
87 0 min
88 0 min
89 0 min
90 0 min
91 0 min
92 0 min
93 0 min
94 0 min
95 0 min
96 0 min
97 0 min
98 0 min
99 0 min
100 0 min
101 0 min
102 0 min
103 0 min
104 0 min
105 0 min
106 0 min
107 0 min
108 0 min
109 0 min
110 0 min
111 0 min
112 0 min
113 0 min
114 0 min
115 0 min
116 0 min
117 0 min
118 0 min
119 0 min
120 0 min
121 0 min
122 0 min
123 0 min
124 0 min
125 0 min
126 0 min
127 0 min
128 0 min
129 0 min
130 0 min
131 0 min
132 0 min
133 0 min
134 0 min
135 0 min
136 0 min
137 0 min
138 0 min
139 0 min
140 0 min
141 0 min
142 0 min
143 0 min
144 0 min
145 0 min
146 0 min
147 0 min
148 0 min
149 0 min
150 0 min
151 0 min
152 0 min
153 0 min
154 0 min
155 0 min
156 0 min
157 0 min
158 0 min
159 0 min
160 0 min
161 0 min
162 0 min
163 0 min
164 0 min
165 0 min
166 0 min
167 0 min
168 0 min
169 0 min
170 0 min
171 0 min
172 0 min
173 0 min
174 0 min
175 0 min
176 0 min
177 0 min
178 0 min
179 0 min
180 0 min
181 0 min
182 0 min
183 0 min
184 0 min
185 0 min
186 0 min
187 0 min
188 0 min
189 0 min
190 0 min
191 0 min
192 0 min
193 0 min
194 0 min
195 0 min
196 0 min
197 0 min
198 0 min
199 0 min
200 0 min
201 0 min
202 0 min
203 0 min
204 0 min
205 0 min
206 0 min
207 0 min
208 0 min
209 0 min
210 0 min
211 0 min
212 0 min
213 0 min
214 0 min
215 0 min
216 0 min
217 0 min
218 0 min
219 0 min
220 0 min
221 0 min
222 0 min
223 0 min
224 0 min
225 0 min
226 0 min
227 0 min
228 0 min
229 0 min
230 0 min
231 0 min
232 0 min
233 0 min
234 0 min
235 0 min
236 0 min
237 0 min
238 0 min
239 0 min
240 0 min
241 0 min
242 0 min
243 0 min
244 0 min
245 0 min
246 0 min
247 0 min
248 0 min
249 0 min
250 0 min
251 0 min
252 0 min
253 0 min
254 0 min
255 0 min
256 0 min
257 0 min
258 0 min
259 0 min
260 0 min
261 0 min
262 0 min
263 0 min
264 0 min
265 0 min
266 0 min
267 0 min
268 0 min
269 0 min
270 0 min
271 0 min
272 0 min
273 0 min
274 0 min
275 0 min
276 0 min
277 0 min
278 0 min
279 0 min
280 0 min
281 0 min
282 0 min
283 0 min
284 0 min
285 0 min
286 0 min
287 0 min
288 0 min
289 0 min
290 0 min
291 0 min
292 0 min
293 0 min
294 0 min
295 0 min
296 0 min
297 0 min
298 0 min
299 0 min
300 0 min
301 0 min
302 0 min
303 0 min
304 0 min
305 0 min
306 0 min
307 0 min
308 0 min
309 0 min
310 0 min
311 0 min
312 0 min
313 0 min
314 0 min
315 0 min
316 0 min
317 0 min
318 0 min
319 0 min
320 0 min
321 0 min
322 0 min
323 0 min
324 0 min
325 0 min
326 0 min
327 0 min
328 0 min
329 0 min
330 0 min
331 0 min
332 0 min
333 0 min
334 0 min
335 0 min
336 0 min
337 0 min
338 0 min
339 0 min
340 0 min
341 0 min
342 0 min
343 0 min
344 0 min
345 0 min
346 0 min
347 0 min
348 0 min
349 0 min
};

\begin{figure}
	\centering
	\pgfplotsset{every axis legend/.append style={
		at={(1.05,0.5)},
		anchor=west}}
	\begin{tikzpicture}
		\begin{axis}[
			xlabel=Cycles passed,
			ylabel=Fitness,
			scatter/classes={
				max={mark=square*,blue},
				med={mark=square*,red},
				min={mark=square*,green}
				}
            ]
            \begin{figure}
	\centering
	\pgfplotsset{every axis legend/.append style={
		at={(1.05,0.5)},
		anchor=west}}
	\begin{tikzpicture}
		\begin{axis}[
			xlabel=Cycles passed,
			ylabel=Fitness,
			scatter/classes={
				max={mark=square*,blue},
				med={mark=square*,red},
				min={mark=square*,green}
				}
            ]
            \begin{figure}
	\centering
	\pgfplotsset{every axis legend/.append style={
		at={(1.05,0.5)},
		anchor=west}}
	\begin{tikzpicture}
		\begin{axis}[
			xlabel=Cycles passed,
			ylabel=Fitness,
			scatter/classes={
				max={mark=square*,blue},
				med={mark=square*,red},
				min={mark=square*,green}
				}
            ]
            \input{data/tex/stats/balanced40.tex}
			\addlegendentry{Maximum}
			\addlegendentry{Median}
			\addlegendentry{Minimum}
		\end{axis}
	\end{tikzpicture}
	\caption{Detail of genetic algorithm on balanced data with 40 elements in set}
\label{plot:genProfile40}
\end{figure}

			\addlegendentry{Maximum}
			\addlegendentry{Median}
			\addlegendentry{Minimum}
		\end{axis}
	\end{tikzpicture}
	\caption{Detail of genetic algorithm on balanced data with 40 elements in set}
\label{plot:genProfile40}
\end{figure}

			\addlegendentry{Maximum}
			\addlegendentry{Median}
			\addlegendentry{Minimum}
		\end{axis}
	\end{tikzpicture}
	\caption{Detail of genetic algorithm on balanced data with 40 elements in set}
\label{plot:genProfile40}
\end{figure}


\addplot+[scatter,only marks,scatter src=explicit symbolic]table[meta=label] {
x y label
0 495452 max
1 495452 max
2 495452 max
3 495452 max
4 522436 max
5 522436 max
6 522436 max
7 522436 max
8 522436 max
9 522436 max
10 522436 max
11 544541 max
12 549889 max
13 549889 max
14 551856 max
15 551856 max
16 562210 max
17 562210 max
18 562210 max
19 562210 max
20 562210 max
21 562210 max
22 562210 max
23 594678 max
24 594678 max
25 594678 max
26 594678 max
27 594678 max
28 594678 max
29 594678 max
30 594678 max
31 594678 max
32 594678 max
33 594678 max
34 594678 max
35 594678 max
36 594678 max
37 594678 max
38 594678 max
39 594678 max
40 594678 max
41 603542 max
42 603542 max
43 603542 max
44 603542 max
45 603542 max
46 603542 max
47 603542 max
48 603542 max
49 603542 max
50 603542 max
51 603542 max
52 603542 max
53 603542 max
54 603542 max
55 603542 max
56 603542 max
57 603542 max
58 603542 max
59 603542 max
60 606970 max
61 606970 max
62 606970 max
63 606970 max
64 606970 max
65 606970 max
66 606970 max
67 606970 max
68 606970 max
69 606970 max
70 606970 max
71 606970 max
72 606970 max
73 606970 max
74 606970 max
75 606970 max
76 606970 max
77 606970 max
78 606970 max
79 606970 max
80 606970 max
81 606970 max
82 606970 max
83 606970 max
84 606970 max
85 606970 max
86 611199 max
87 611199 max
88 611199 max
89 611199 max
90 611199 max
91 611199 max
92 611199 max
93 611199 max
94 611199 max
95 611199 max
96 611199 max
97 611199 max
98 611199 max
99 611199 max
100 611199 max
101 611199 max
102 611199 max
103 611199 max
104 611199 max
105 611199 max
106 611199 max
107 612351 max
108 612351 max
109 612351 max
110 612351 max
111 612351 max
112 612351 max
113 612351 max
114 612351 max
115 612351 max
116 612351 max
117 612351 max
118 612351 max
119 612351 max
120 612351 max
121 612351 max
122 612351 max
123 612351 max
124 612351 max
125 612351 max
126 612351 max
127 612351 max
128 612351 max
129 612351 max
130 612351 max
131 612351 max
132 612351 max
133 612351 max
134 612351 max
135 612351 max
136 612351 max
137 612351 max
138 612351 max
139 612351 max
140 612351 max
141 612351 max
142 612351 max
143 612351 max
144 612351 max
145 612351 max
146 612351 max
147 612351 max
148 612351 max
149 612351 max
150 612351 max
151 612351 max
152 612351 max
153 612351 max
154 612351 max
155 612351 max
156 612351 max
157 612351 max
158 612351 max
159 612351 max
160 612351 max
161 612351 max
162 612351 max
163 612351 max
164 612351 max
165 612351 max
166 612351 max
167 612351 max
168 612351 max
169 612351 max
170 612351 max
171 612351 max
172 612351 max
173 612351 max
174 612351 max
175 612351 max
176 612351 max
177 612351 max
178 612351 max
179 612351 max
180 612351 max
181 612351 max
182 612351 max
183 612351 max
184 612351 max
185 612351 max
186 612351 max
187 612351 max
188 612351 max
189 612351 max
190 612351 max
191 612351 max
192 612351 max
193 612351 max
194 612351 max
195 612351 max
196 612351 max
197 612351 max
198 612351 max
199 612351 max
200 612351 max
201 612351 max
202 612351 max
203 612351 max
204 612351 max
205 612351 max
206 612351 max
207 612351 max
208 612351 max
209 612351 max
210 612351 max
211 612351 max
212 612351 max
213 612351 max
214 612351 max
215 612351 max
216 612351 max
217 612351 max
218 612351 max
219 612351 max
220 612351 max
221 622619 max
222 622619 max
223 622619 max
224 622619 max
225 622619 max
226 622619 max
227 622619 max
228 622619 max
229 622619 max
230 622619 max
231 622619 max
232 622619 max
233 622619 max
234 622619 max
235 622619 max
236 622619 max
237 622619 max
238 622619 max
239 622619 max
240 622619 max
241 622619 max
242 622619 max
243 622619 max
244 622619 max
245 622619 max
246 622619 max
247 622619 max
248 622619 max
249 622619 max
250 622619 max
251 622619 max
252 622619 max
253 622619 max
254 622619 max
255 622619 max
256 622619 max
257 622619 max
258 622619 max
259 622619 max
260 622619 max
261 622619 max
262 622619 max
263 622619 max
264 622619 max
265 622619 max
266 622619 max
267 622619 max
268 622619 max
269 622619 max
270 622619 max
271 622619 max
272 622619 max
273 622619 max
274 622619 max
275 622619 max
276 622619 max
277 622619 max
278 622619 max
279 622619 max
280 622619 max
281 622619 max
282 622619 max
283 622619 max
284 622619 max
285 622619 max
286 622619 max
287 622619 max
288 622619 max
289 622619 max
290 622619 max
291 622619 max
292 622619 max
293 622619 max
294 622619 max
295 622619 max
296 622619 max
297 622619 max
298 622619 max
299 622619 max
300 622619 max
301 622619 max
302 622619 max
303 622619 max
304 622619 max
305 622619 max
306 622619 max
307 622619 max
308 622619 max
309 622619 max
310 622619 max
311 622619 max
312 622619 max
313 622619 max
314 622619 max
315 622619 max
316 622619 max
317 622619 max
318 622619 max
319 622619 max
320 622619 max
321 622619 max
322 622619 max
323 622619 max
324 622619 max
325 622619 max
326 622619 max
327 622619 max
328 622619 max
329 622619 max
330 622619 max
331 622619 max
332 622619 max
333 622619 max
334 622619 max
335 622619 max
336 622619 max
337 622619 max
338 622619 max
339 622619 max
340 622619 max
341 622619 max
342 622619 max
343 622619 max
344 622619 max
345 622619 max
346 622619 max
347 622619 max
348 622619 max
349 622619 max
350 622619 max
351 622619 max
352 622619 max
353 622619 max
354 622619 max
355 622619 max
356 622619 max
357 622619 max
358 622619 max
359 622619 max
360 622619 max
361 622619 max
362 622619 max
363 622619 max
364 622619 max
365 622619 max
366 622619 max
367 622619 max
368 622619 max
369 622619 max
370 622619 max
371 622619 max
372 622619 max
373 622619 max
374 622619 max
375 622619 max
376 622619 max
377 622619 max
378 622619 max
379 622619 max
380 622619 max
381 622619 max
382 622619 max
383 622619 max
384 622619 max
385 622619 max
386 622619 max
387 622619 max
388 622619 max
389 622619 max
390 622619 max
391 622619 max
392 622619 max
393 622619 max
394 622619 max
395 622619 max
396 622619 max
397 622619 max
398 622619 max
399 622619 max
400 622619 max
401 622619 max
402 622619 max
403 622619 max
404 622619 max
405 622619 max
406 622619 max
407 622619 max
408 622619 max
409 622619 max
410 622619 max
411 622619 max
412 622619 max
413 622619 max
414 622619 max
415 622619 max
416 622619 max
417 622619 max
418 622619 max
419 622619 max
420 622619 max
421 622619 max
422 622619 max
423 622619 max
424 622619 max
425 622619 max
426 622619 max
427 622619 max
428 622619 max
429 622619 max
430 622619 max
431 622619 max
432 622619 max
433 622619 max
434 622619 max
435 622619 max
436 622619 max
437 622619 max
438 622619 max
439 622619 max
440 622619 max
441 622619 max
442 622619 max
443 622619 max
444 622619 max
445 622619 max
446 622619 max
447 622619 max
448 622619 max
449 622619 max
};
\addplot+[scatter,only marks,scatter src=explicit symbolic]table[meta=label] {
x y label
0 318117 med
1 338322 med
2 370866 med
3 370105 med
4 402694 med
5 400225 med
6 407357 med
7 413212 med
8 382476 med
9 399637 med
10 329888 med
11 420753 med
12 426815 med
13 0 med
14 412058 med
15 425087 med
16 435370 med
17 418082 med
18 398752 med
19 408432 med
20 409051 med
21 432988 med
22 0 med
23 414203 med
24 462617 med
25 448663 med
26 401600 med
27 471219 med
28 409908 med
29 447657 med
30 437937 med
31 438279 med
32 438109 med
33 442685 med
34 437810 med
35 463765 med
36 402991 med
37 468950 med
38 472605 med
39 412651 med
40 410787 med
41 420263 med
42 444856 med
43 491258 med
44 400218 med
45 457721 med
46 347232 med
47 461978 med
48 482698 med
49 0 med
50 461540 med
51 348978 med
52 486586 med
53 490447 med
54 457997 med
55 0 med
56 362312 med
57 481868 med
58 470241 med
59 0 med
60 459559 med
61 420073 med
62 460718 med
63 449804 med
64 458279 med
65 511292 med
66 451729 med
67 494251 med
68 438676 med
69 343350 med
70 487820 med
71 441284 med
72 449551 med
73 453671 med
74 385522 med
75 0 med
76 480760 med
77 409354 med
78 470746 med
79 465393 med
80 441686 med
81 482099 med
82 459000 med
83 439071 med
84 455780 med
85 375485 med
86 396378 med
87 0 med
88 462962 med
89 0 med
90 0 med
91 478438 med
92 495902 med
93 462229 med
94 490298 med
95 490551 med
96 399093 med
97 451204 med
98 450070 med
99 0 med
100 413594 med
101 416793 med
102 431172 med
103 435615 med
104 453991 med
105 421757 med
106 493577 med
107 480955 med
108 0 med
109 335125 med
110 480587 med
111 472127 med
112 496229 med
113 471538 med
114 417541 med
115 466080 med
116 0 med
117 450258 med
118 419241 med
119 480294 med
120 442677 med
121 497287 med
122 436014 med
123 0 med
124 458693 med
125 493865 med
126 495318 med
127 441912 med
128 407738 med
129 387994 med
130 445249 med
131 483690 med
132 446071 med
133 442285 med
134 0 med
135 443923 med
136 485248 med
137 464773 med
138 450402 med
139 439005 med
140 469434 med
141 425698 med
142 457651 med
143 426056 med
144 439059 med
145 0 med
146 428235 med
147 410309 med
148 484617 med
149 494537 med
150 452866 med
151 476084 med
152 464707 med
153 408921 med
154 448497 med
155 343257 med
156 0 med
157 437865 med
158 427404 med
159 487879 med
160 481790 med
161 483568 med
162 426010 med
163 454857 med
164 476393 med
165 434535 med
166 437580 med
167 475308 med
168 395481 med
169 422289 med
170 477593 med
171 493160 med
172 415975 med
173 440603 med
174 463650 med
175 443510 med
176 424161 med
177 428577 med
178 386742 med
179 437731 med
180 480540 med
181 447165 med
182 0 med
183 403338 med
184 464599 med
185 492182 med
186 478578 med
187 471068 med
188 431594 med
189 438440 med
190 459178 med
191 476815 med
192 473860 med
193 427793 med
194 419388 med
195 0 med
196 486603 med
197 465866 med
198 403058 med
199 0 med
200 455389 med
201 447671 med
202 454746 med
203 0 med
204 0 med
205 447268 med
206 0 med
207 423608 med
208 423026 med
209 0 med
210 441774 med
211 439003 med
212 408489 med
213 0 med
214 395652 med
215 442797 med
216 466451 med
217 444322 med
218 460148 med
219 482290 med
220 0 med
221 474202 med
222 456824 med
223 370512 med
224 0 med
225 397364 med
226 482153 med
227 502176 med
228 441255 med
229 490158 med
230 481281 med
231 443975 med
232 446406 med
233 460843 med
234 398863 med
235 435446 med
236 0 med
237 446456 med
238 0 med
239 420821 med
240 473138 med
241 473175 med
242 486787 med
243 422959 med
244 0 med
245 463680 med
246 432709 med
247 486003 med
248 469599 med
249 0 med
250 367997 med
251 0 med
252 0 med
253 472498 med
254 455063 med
255 449625 med
256 494274 med
257 466531 med
258 0 med
259 443962 med
260 0 med
261 463061 med
262 0 med
263 471122 med
264 442071 med
265 433738 med
266 393150 med
267 450080 med
268 488426 med
269 401677 med
270 0 med
271 338473 med
272 448280 med
273 475063 med
274 492776 med
275 481346 med
276 469877 med
277 473244 med
278 478751 med
279 459564 med
280 0 med
281 488828 med
282 453176 med
283 438657 med
284 452885 med
285 421877 med
286 0 med
287 398538 med
288 493129 med
289 487784 med
290 0 med
291 0 med
292 0 med
293 451121 med
294 452508 med
295 442983 med
296 434532 med
297 0 med
298 0 med
299 417324 med
300 432563 med
301 483841 med
302 475814 med
303 468170 med
304 329062 med
305 0 med
306 436010 med
307 473355 med
308 462482 med
309 474231 med
310 507069 med
311 449688 med
312 433466 med
313 452087 med
314 376927 med
315 470004 med
316 386904 med
317 472257 med
318 436072 med
319 378562 med
320 431725 med
321 0 med
322 0 med
323 443855 med
324 0 med
325 368990 med
326 469958 med
327 467902 med
328 418486 med
329 480401 med
330 472339 med
331 449726 med
332 430480 med
333 370267 med
334 0 med
335 413693 med
336 448124 med
337 339554 med
338 0 med
339 459959 med
340 481124 med
341 474650 med
342 477109 med
343 486521 med
344 403502 med
345 0 med
346 474565 med
347 472846 med
348 479080 med
349 455438 med
350 510077 med
351 487212 med
352 409222 med
353 433239 med
354 364598 med
355 505538 med
356 464387 med
357 421161 med
358 0 med
359 0 med
360 0 med
361 0 med
362 397428 med
363 0 med
364 0 med
365 428587 med
366 0 med
367 420187 med
368 486688 med
369 489359 med
370 450713 med
371 400878 med
372 429676 med
373 448850 med
374 500124 med
375 443882 med
376 481566 med
377 468995 med
378 403565 med
379 437567 med
380 0 med
381 401121 med
382 448024 med
383 0 med
384 0 med
385 467369 med
386 469835 med
387 0 med
388 472881 med
389 410059 med
390 426725 med
391 482948 med
392 483934 med
393 448315 med
394 396684 med
395 466997 med
396 447834 med
397 437117 med
398 0 med
399 0 med
400 474178 med
401 471170 med
402 477419 med
403 503704 med
404 447518 med
405 408430 med
406 467548 med
407 421917 med
408 456970 med
409 475824 med
410 437917 med
411 409301 med
412 450432 med
413 472333 med
414 436751 med
415 432828 med
416 420737 med
417 364234 med
418 0 med
419 0 med
420 466323 med
421 0 med
422 452405 med
423 465246 med
424 469153 med
425 453636 med
426 489832 med
427 0 med
428 466420 med
429 455989 med
430 348523 med
431 448131 med
432 359714 med
433 0 med
434 0 med
435 468989 med
436 0 med
437 449667 med
438 426906 med
439 0 med
440 454552 med
441 357371 med
442 455903 med
443 495684 med
444 467857 med
445 515027 med
446 459679 med
447 447297 med
448 0 med
449 422815 med
};
\addplot+[scatter,only marks,scatter src=explicit symbolic]table[meta=label] {
x y label
0 0 min
1 0 min
2 0 min
3 0 min
4 0 min
5 0 min
6 0 min
7 0 min
8 0 min
9 0 min
10 0 min
11 0 min
12 0 min
13 0 min
14 0 min
15 0 min
16 0 min
17 0 min
18 0 min
19 0 min
20 0 min
21 0 min
22 0 min
23 0 min
24 0 min
25 0 min
26 0 min
27 0 min
28 0 min
29 0 min
30 0 min
31 0 min
32 0 min
33 0 min
34 0 min
35 0 min
36 0 min
37 0 min
38 0 min
39 0 min
40 0 min
41 0 min
42 0 min
43 0 min
44 0 min
45 0 min
46 0 min
47 0 min
48 0 min
49 0 min
50 0 min
51 0 min
52 0 min
53 0 min
54 0 min
55 0 min
56 0 min
57 0 min
58 0 min
59 0 min
60 0 min
61 0 min
62 0 min
63 0 min
64 0 min
65 0 min
66 0 min
67 0 min
68 0 min
69 0 min
70 0 min
71 0 min
72 0 min
73 0 min
74 0 min
75 0 min
76 0 min
77 0 min
78 0 min
79 0 min
80 0 min
81 0 min
82 0 min
83 0 min
84 0 min
85 0 min
86 0 min
87 0 min
88 0 min
89 0 min
90 0 min
91 0 min
92 0 min
93 0 min
94 0 min
95 0 min
96 0 min
97 0 min
98 0 min
99 0 min
100 0 min
101 0 min
102 0 min
103 0 min
104 0 min
105 0 min
106 0 min
107 0 min
108 0 min
109 0 min
110 0 min
111 0 min
112 0 min
113 0 min
114 0 min
115 0 min
116 0 min
117 0 min
118 0 min
119 0 min
120 0 min
121 0 min
122 0 min
123 0 min
124 0 min
125 0 min
126 0 min
127 0 min
128 0 min
129 0 min
130 0 min
131 0 min
132 0 min
133 0 min
134 0 min
135 0 min
136 0 min
137 0 min
138 0 min
139 0 min
140 0 min
141 0 min
142 0 min
143 0 min
144 0 min
145 0 min
146 0 min
147 0 min
148 0 min
149 0 min
150 0 min
151 0 min
152 0 min
153 0 min
154 0 min
155 0 min
156 0 min
157 0 min
158 0 min
159 0 min
160 0 min
161 0 min
162 0 min
163 0 min
164 0 min
165 0 min
166 0 min
167 0 min
168 0 min
169 0 min
170 0 min
171 0 min
172 0 min
173 0 min
174 0 min
175 0 min
176 0 min
177 0 min
178 0 min
179 0 min
180 0 min
181 0 min
182 0 min
183 0 min
184 0 min
185 0 min
186 0 min
187 0 min
188 0 min
189 0 min
190 0 min
191 0 min
192 0 min
193 0 min
194 0 min
195 0 min
196 0 min
197 0 min
198 0 min
199 0 min
200 0 min
201 0 min
202 0 min
203 0 min
204 0 min
205 0 min
206 0 min
207 0 min
208 0 min
209 0 min
210 0 min
211 0 min
212 0 min
213 0 min
214 0 min
215 0 min
216 0 min
217 0 min
218 0 min
219 0 min
220 0 min
221 0 min
222 0 min
223 0 min
224 0 min
225 0 min
226 0 min
227 0 min
228 0 min
229 0 min
230 0 min
231 0 min
232 0 min
233 0 min
234 0 min
235 0 min
236 0 min
237 0 min
238 0 min
239 0 min
240 0 min
241 0 min
242 0 min
243 0 min
244 0 min
245 0 min
246 0 min
247 0 min
248 0 min
249 0 min
250 0 min
251 0 min
252 0 min
253 0 min
254 0 min
255 0 min
256 0 min
257 0 min
258 0 min
259 0 min
260 0 min
261 0 min
262 0 min
263 0 min
264 0 min
265 0 min
266 0 min
267 0 min
268 0 min
269 0 min
270 0 min
271 0 min
272 0 min
273 0 min
274 0 min
275 0 min
276 0 min
277 0 min
278 0 min
279 0 min
280 0 min
281 0 min
282 0 min
283 0 min
284 0 min
285 0 min
286 0 min
287 0 min
288 0 min
289 0 min
290 0 min
291 0 min
292 0 min
293 0 min
294 0 min
295 0 min
296 0 min
297 0 min
298 0 min
299 0 min
300 0 min
301 0 min
302 0 min
303 0 min
304 0 min
305 0 min
306 0 min
307 0 min
308 0 min
309 0 min
310 0 min
311 0 min
312 0 min
313 0 min
314 0 min
315 0 min
316 0 min
317 0 min
318 0 min
319 0 min
320 0 min
321 0 min
322 0 min
323 0 min
324 0 min
325 0 min
326 0 min
327 0 min
328 0 min
329 0 min
330 0 min
331 0 min
332 0 min
333 0 min
334 0 min
335 0 min
336 0 min
337 0 min
338 0 min
339 0 min
340 0 min
341 0 min
342 0 min
343 0 min
344 0 min
345 0 min
346 0 min
347 0 min
348 0 min
349 0 min
350 0 min
351 0 min
352 0 min
353 0 min
354 0 min
355 0 min
356 0 min
357 0 min
358 0 min
359 0 min
360 0 min
361 0 min
362 0 min
363 0 min
364 0 min
365 0 min
366 0 min
367 0 min
368 0 min
369 0 min
370 0 min
371 0 min
372 0 min
373 0 min
374 0 min
375 0 min
376 0 min
377 0 min
378 0 min
379 0 min
380 0 min
381 0 min
382 0 min
383 0 min
384 0 min
385 0 min
386 0 min
387 0 min
388 0 min
389 0 min
390 0 min
391 0 min
392 0 min
393 0 min
394 0 min
395 0 min
396 0 min
397 0 min
398 0 min
399 0 min
400 0 min
401 0 min
402 0 min
403 0 min
404 0 min
405 0 min
406 0 min
407 0 min
408 0 min
409 0 min
410 0 min
411 0 min
412 0 min
413 0 min
414 0 min
415 0 min
416 0 min
417 0 min
418 0 min
419 0 min
420 0 min
421 0 min
422 0 min
423 0 min
424 0 min
425 0 min
426 0 min
427 0 min
428 0 min
429 0 min
430 0 min
431 0 min
432 0 min
433 0 min
434 0 min
435 0 min
436 0 min
437 0 min
438 0 min
439 0 min
440 0 min
441 0 min
442 0 min
443 0 min
444 0 min
445 0 min
446 0 min
447 0 min
448 0 min
449 0 min
};




\addplot+[scatter,only marks,scatter src=explicit symbolic]table[meta=label] {
x y label
0 272962 max
1 294172 max
2 304979 max
3 304979 max
4 322377 max
5 325684 max
6 344312 max
7 355718 max
8 355718 max
9 355718 max
10 365209 max
11 365209 max
12 365209 max
13 365209 max
14 372043 max
15 372043 max
16 372043 max
17 372043 max
18 372043 max
19 372043 max
20 375070 max
21 376222 max
22 376222 max
23 376269 max
24 376269 max
25 376269 max
26 376269 max
27 376269 max
28 376269 max
29 376269 max
30 376269 max
31 376269 max
32 376269 max
33 376269 max
34 376269 max
35 376269 max
36 376269 max
37 376269 max
38 376269 max
39 376269 max
40 376269 max
41 376269 max
42 376269 max
43 376269 max
44 376269 max
45 376269 max
46 376269 max
47 376269 max
48 379206 max
49 379206 max
50 379206 max
51 379206 max
52 379206 max
53 379206 max
54 379206 max
55 379206 max
56 379206 max
57 379206 max
58 379206 max
59 379206 max
60 379206 max
61 379206 max
62 379206 max
63 379206 max
64 379206 max
65 379206 max
66 379206 max
67 379206 max
68 379206 max
69 379206 max
70 379206 max
71 379206 max
72 379206 max
73 379206 max
74 381283 max
75 381283 max
76 381283 max
77 381283 max
78 381283 max
79 381283 max
80 381283 max
81 381283 max
82 381283 max
83 381283 max
84 381283 max
85 381283 max
86 381283 max
87 381283 max
88 381283 max
89 381283 max
90 381283 max
91 381283 max
92 381283 max
93 381283 max
94 381283 max
95 381283 max
96 381283 max
97 381283 max
98 381283 max
99 381283 max
100 381283 max
101 381283 max
102 381283 max
103 381283 max
104 381283 max
105 381283 max
106 381283 max
107 381283 max
108 381283 max
109 381283 max
110 381283 max
111 381283 max
112 381283 max
113 381283 max
114 381283 max
115 381283 max
116 381283 max
117 381283 max
118 381283 max
119 381283 max
120 381283 max
121 381283 max
122 381283 max
123 381283 max
124 381283 max
125 381283 max
126 381283 max
127 381283 max
128 381283 max
129 381283 max
130 381283 max
131 381283 max
132 381283 max
133 381283 max
134 381283 max
135 381283 max
136 381283 max
137 381283 max
138 381283 max
139 381283 max
140 381283 max
141 381283 max
142 381283 max
143 381283 max
144 381283 max
145 381283 max
146 381283 max
147 381283 max
148 381283 max
149 381283 max
150 381283 max
151 381283 max
152 381283 max
153 381283 max
154 381283 max
155 381283 max
156 381283 max
157 381283 max
158 381283 max
159 381283 max
160 381283 max
161 381283 max
162 381283 max
163 381283 max
164 381283 max
165 381283 max
166 381283 max
167 381283 max
168 381283 max
169 381283 max
170 381283 max
171 381283 max
172 381283 max
173 381283 max
174 381283 max
175 381283 max
176 381283 max
177 381283 max
178 381283 max
179 381283 max
180 381283 max
181 381283 max
182 381283 max
183 381283 max
184 381283 max
185 381283 max
186 381283 max
187 381283 max
188 381283 max
189 381283 max
190 381283 max
191 381283 max
192 381283 max
193 381283 max
194 381283 max
195 381283 max
196 381283 max
197 381283 max
198 381283 max
199 381283 max
200 381283 max
201 381283 max
202 381283 max
203 381283 max
204 381283 max
205 381283 max
206 381283 max
207 381283 max
208 381283 max
209 381283 max
210 381283 max
211 381283 max
212 381283 max
213 381283 max
214 381283 max
215 381283 max
216 381283 max
217 381283 max
218 381283 max
219 381283 max
220 381283 max
221 381283 max
222 381283 max
223 381283 max
224 381283 max
225 381283 max
226 381283 max
227 381283 max
228 381283 max
229 381283 max
230 381283 max
231 381283 max
232 381283 max
233 381283 max
234 381283 max
235 381283 max
236 381283 max
237 381283 max
238 381283 max
239 381283 max
240 381283 max
241 381283 max
242 381283 max
243 381283 max
244 381283 max
245 381283 max
246 381283 max
247 381283 max
248 381283 max
249 381283 max
250 381283 max
251 381283 max
252 381283 max
253 381283 max
254 381283 max
255 381283 max
256 381283 max
257 381283 max
258 381283 max
259 381283 max
260 381283 max
261 381283 max
262 381283 max
263 381283 max
264 381283 max
265 381283 max
266 381283 max
267 381283 max
268 381283 max
269 381283 max
270 381283 max
271 381283 max
272 381283 max
273 381283 max
274 381283 max
275 381283 max
276 381283 max
277 381283 max
278 381283 max
279 381283 max
280 381283 max
281 381283 max
282 381283 max
283 381283 max
284 381283 max
285 381283 max
286 381283 max
287 381283 max
288 381283 max
289 381283 max
290 381283 max
291 381283 max
292 381283 max
293 381283 max
294 381283 max
295 381283 max
296 381283 max
297 381283 max
298 381283 max
299 381283 max
};
\addplot+[scatter,only marks,scatter src=explicit symbolic]table[meta=label] {
x y label
0 177071 med
1 203689 med
2 0 med
3 222387 med
4 246048 med
5 260727 med
6 291266 med
7 304652 med
8 316684 med
9 298856 med
10 310895 med
11 325529 med
12 325482 med
13 321731 med
14 331839 med
15 328313 med
16 334896 med
17 329355 med
18 323019 med
19 333846 med
20 0 med
21 316644 med
22 0 med
23 339992 med
24 342835 med
25 342805 med
26 342835 med
27 340190 med
28 340704 med
29 324668 med
30 337383 med
31 349165 med
32 344293 med
33 324633 med
34 304581 med
35 340190 med
36 325480 med
37 0 med
38 0 med
39 348192 med
40 344691 med
41 349100 med
42 343403 med
43 345092 med
44 334234 med
45 340273 med
46 319303 med
47 0 med
48 0 med
49 0 med
50 325604 med
51 312062 med
52 346244 med
53 319819 med
54 0 med
55 331326 med
56 331272 med
57 312549 med
58 340870 med
59 333786 med
60 339221 med
61 337450 med
62 306591 med
63 0 med
64 326695 med
65 336640 med
66 326920 med
67 324321 med
68 318680 med
69 322584 med
70 340262 med
71 330146 med
72 335143 med
73 325497 med
74 338768 med
75 333000 med
76 334234 med
77 335436 med
78 334852 med
79 328772 med
80 343356 med
81 355122 med
82 347683 med
83 349834 med
84 350694 med
85 337716 med
86 337774 med
87 344868 med
88 323601 med
89 335175 med
90 337038 med
91 0 med
92 318810 med
93 0 med
94 344138 med
95 335875 med
96 350265 med
97 334234 med
98 332342 med
99 332587 med
100 342070 med
101 321539 med
102 335462 med
103 340572 med
104 348307 med
105 336521 med
106 332861 med
107 314439 med
108 0 med
109 322328 med
110 337864 med
111 336001 med
112 348476 med
113 335931 med
114 330240 med
115 345633 med
116 342224 med
117 345069 med
118 340572 med
119 0 med
120 336436 med
121 341654 med
122 321849 med
123 337716 med
124 345765 med
125 349532 med
126 349834 med
127 309110 med
128 336638 med
129 293592 med
130 328578 med
131 345598 med
132 344135 med
133 346690 med
134 342983 med
135 345939 med
136 360185 med
137 308738 med
138 0 med
139 328471 med
140 337439 med
141 328112 med
142 350516 med
143 337203 med
144 300592 med
145 353476 med
146 288533 med
147 318993 med
148 335942 med
149 348380 med
150 326902 med
151 346917 med
152 345074 med
153 322385 med
154 335366 med
155 0 med
156 319546 med
157 331532 med
158 339869 med
159 357042 med
160 355613 med
161 348380 med
162 330740 med
163 0 med
164 306252 med
165 321888 med
166 332861 med
167 0 med
168 332587 med
169 332732 med
170 343193 med
171 321603 med
172 324959 med
173 336393 med
174 331503 med
175 317276 med
176 0 med
177 0 med
178 0 med
179 334561 med
180 340572 med
181 341789 med
182 352037 med
183 338783 med
184 314294 med
185 337152 med
186 337716 med
187 343656 med
188 324421 med
189 333817 med
190 341789 med
191 324959 med
192 331433 med
193 351258 med
194 332448 med
195 332448 med
196 345759 med
197 350704 med
198 355603 med
199 347163 med
200 346643 med
201 341535 med
202 305861 med
203 333351 med
204 358979 med
205 341789 med
206 320440 med
207 345759 med
208 294483 med
209 349178 med
210 340906 med
211 343403 med
212 343403 med
213 335476 med
214 340084 med
215 327524 med
216 330240 med
217 327598 med
218 324959 med
219 0 med
220 318110 med
221 0 med
222 0 med
223 0 med
224 309524 med
225 328231 med
226 344326 med
227 307319 med
228 322800 med
229 342224 med
230 344326 med
231 329465 med
232 327042 med
233 341061 med
234 336701 med
235 336638 med
236 350694 med
237 350056 med
238 333101 med
239 346127 med
240 350056 med
241 356891 med
242 0 med
243 328157 med
244 325194 med
245 331378 med
246 351911 med
247 337800 med
248 318590 med
249 304471 med
250 337976 med
251 321640 med
252 309786 med
253 335116 med
254 331532 med
255 322218 med
256 344483 med
257 345246 med
258 349018 med
259 289567 med
260 331494 med
261 325699 med
262 334798 med
263 340066 med
264 345074 med
265 323150 med
266 327396 med
267 319694 med
268 334694 med
269 340300 med
270 331340 med
271 341109 med
272 353767 med
273 330079 med
274 346484 med
275 350056 med
276 332006 med
277 285002 med
278 321186 med
279 334678 med
280 349067 med
281 336880 med
282 345074 med
283 337152 med
284 334537 med
285 331532 med
286 347838 med
287 341789 med
288 340859 med
289 345759 med
290 338246 med
291 322795 med
292 315702 med
293 330977 med
294 343018 med
295 345882 med
296 352222 med
297 340906 med
298 345765 med
299 327560 med
};
\addplot+[scatter,only marks,scatter src=explicit symbolic]table[meta=label] {
x y label
0 0 min
1 0 min
2 0 min
3 0 min
4 0 min
5 0 min
6 0 min
7 0 min
8 0 min
9 0 min
10 0 min
11 0 min
12 0 min
13 0 min
14 0 min
15 0 min
16 0 min
17 0 min
18 0 min
19 0 min
20 0 min
21 0 min
22 0 min
23 0 min
24 0 min
25 0 min
26 0 min
27 0 min
28 0 min
29 0 min
30 0 min
31 0 min
32 0 min
33 0 min
34 0 min
35 0 min
36 0 min
37 0 min
38 0 min
39 0 min
40 0 min
41 0 min
42 0 min
43 0 min
44 0 min
45 0 min
46 0 min
47 0 min
48 0 min
49 0 min
50 0 min
51 0 min
52 0 min
53 0 min
54 0 min
55 0 min
56 0 min
57 0 min
58 0 min
59 0 min
60 0 min
61 0 min
62 0 min
63 0 min
64 0 min
65 0 min
66 0 min
67 0 min
68 0 min
69 0 min
70 0 min
71 0 min
72 0 min
73 0 min
74 0 min
75 0 min
76 0 min
77 0 min
78 0 min
79 0 min
80 0 min
81 0 min
82 0 min
83 0 min
84 0 min
85 0 min
86 0 min
87 0 min
88 0 min
89 0 min
90 0 min
91 0 min
92 0 min
93 0 min
94 0 min
95 0 min
96 0 min
97 0 min
98 0 min
99 0 min
100 0 min
101 0 min
102 0 min
103 0 min
104 0 min
105 0 min
106 0 min
107 0 min
108 0 min
109 0 min
110 0 min
111 0 min
112 0 min
113 0 min
114 0 min
115 0 min
116 0 min
117 0 min
118 0 min
119 0 min
120 0 min
121 0 min
122 0 min
123 0 min
124 0 min
125 0 min
126 0 min
127 0 min
128 0 min
129 0 min
130 0 min
131 0 min
132 0 min
133 0 min
134 0 min
135 0 min
136 0 min
137 0 min
138 0 min
139 0 min
140 0 min
141 0 min
142 0 min
143 0 min
144 0 min
145 0 min
146 0 min
147 0 min
148 0 min
149 0 min
150 0 min
151 0 min
152 0 min
153 0 min
154 0 min
155 0 min
156 0 min
157 0 min
158 0 min
159 0 min
160 0 min
161 0 min
162 0 min
163 0 min
164 0 min
165 0 min
166 0 min
167 0 min
168 0 min
169 0 min
170 0 min
171 0 min
172 0 min
173 0 min
174 0 min
175 0 min
176 0 min
177 0 min
178 0 min
179 0 min
180 0 min
181 0 min
182 0 min
183 0 min
184 0 min
185 0 min
186 0 min
187 0 min
188 0 min
189 0 min
190 0 min
191 0 min
192 0 min
193 0 min
194 0 min
195 0 min
196 0 min
197 0 min
198 0 min
199 0 min
200 0 min
201 0 min
202 0 min
203 0 min
204 0 min
205 0 min
206 0 min
207 0 min
208 0 min
209 0 min
210 0 min
211 0 min
212 0 min
213 0 min
214 0 min
215 0 min
216 0 min
217 0 min
218 0 min
219 0 min
220 0 min
221 0 min
222 0 min
223 0 min
224 0 min
225 0 min
226 0 min
227 0 min
228 0 min
229 0 min
230 0 min
231 0 min
232 0 min
233 0 min
234 0 min
235 0 min
236 0 min
237 0 min
238 0 min
239 0 min
240 0 min
241 0 min
242 0 min
243 0 min
244 0 min
245 0 min
246 0 min
247 0 min
248 0 min
249 0 min
250 0 min
251 0 min
252 0 min
253 0 min
254 0 min
255 0 min
256 0 min
257 0 min
258 0 min
259 0 min
260 0 min
261 0 min
262 0 min
263 0 min
264 0 min
265 0 min
266 0 min
267 0 min
268 0 min
269 0 min
270 0 min
271 0 min
272 0 min
273 0 min
274 0 min
275 0 min
276 0 min
277 0 min
278 0 min
279 0 min
280 0 min
281 0 min
282 0 min
283 0 min
284 0 min
285 0 min
286 0 min
287 0 min
288 0 min
289 0 min
290 0 min
291 0 min
292 0 min
293 0 min
294 0 min
295 0 min
296 0 min
297 0 min
298 0 min
299 0 min
};

\begin{figure}
	\centering
	\pgfplotsset{every axis legend/.append style={
		at={(1.05,0.5)},
		anchor=west}}
	\begin{tikzpicture}
		\begin{axis}[
			xlabel=Cycles passed,
			ylabel=Fitness,
			scatter/classes={
				max={mark=square*,blue},
				med={mark=square*,red},
				min={mark=square*,green}
				}
            ]
            \begin{figure}
	\centering
	\pgfplotsset{every axis legend/.append style={
		at={(1.05,0.5)},
		anchor=west}}
	\begin{tikzpicture}
		\begin{axis}[
			xlabel=Cycles passed,
			ylabel=Fitness,
			scatter/classes={
				max={mark=square*,blue},
				med={mark=square*,red},
				min={mark=square*,green}
				}
            ]
            \begin{figure}
	\centering
	\pgfplotsset{every axis legend/.append style={
		at={(1.05,0.5)},
		anchor=west}}
	\begin{tikzpicture}
		\begin{axis}[
			xlabel=Cycles passed,
			ylabel=Fitness,
			scatter/classes={
				max={mark=square*,blue},
				med={mark=square*,red},
				min={mark=square*,green}
				}
            ]
            \input{data/tex/stats/balanced30-mutation32.tex}
			\addlegendentry{Maximum}
			\addlegendentry{Median}
			\addlegendentry{Minimum}
		\end{axis}
	\end{tikzpicture}
	\caption{Detail of genetic algorithm on balanced data with 30 elements in set and mutation chance of 1/32}
\label{plot:genProfile30-mutation32}
\end{figure}

			\addlegendentry{Maximum}
			\addlegendentry{Median}
			\addlegendentry{Minimum}
		\end{axis}
	\end{tikzpicture}
	\caption{Detail of genetic algorithm on balanced data with 30 elements in set and mutation chance of 1/32}
\label{plot:genProfile30-mutation32}
\end{figure}

			\addlegendentry{Maximum}
			\addlegendentry{Median}
			\addlegendentry{Minimum}
		\end{axis}
	\end{tikzpicture}
	\caption{Detail of genetic algorithm on balanced data with 30 elements in set and mutation chance of 1/32}
\label{plot:genProfile30-mutation32}
\end{figure}

\begin{figure}
	\centering
	\pgfplotsset{every axis legend/.append style={
		at={(1.05,0.5)},
		anchor=west}}
	\begin{tikzpicture}
		\begin{axis}[
			xlabel=Cycles passed,
			ylabel=Fitness,
			scatter/classes={
				max={mark=square*,blue},
				med={mark=square*,red},
				min={mark=square*,green}
				}
            ]
            \begin{figure}
	\centering
	\pgfplotsset{every axis legend/.append style={
		at={(1.05,0.5)},
		anchor=west}}
	\begin{tikzpicture}
		\begin{axis}[
			xlabel=Cycles passed,
			ylabel=Fitness,
			scatter/classes={
				max={mark=square*,blue},
				med={mark=square*,red},
				min={mark=square*,green}
				}
            ]
            \begin{figure}
	\centering
	\pgfplotsset{every axis legend/.append style={
		at={(1.05,0.5)},
		anchor=west}}
	\begin{tikzpicture}
		\begin{axis}[
			xlabel=Cycles passed,
			ylabel=Fitness,
			scatter/classes={
				max={mark=square*,blue},
				med={mark=square*,red},
				min={mark=square*,green}
				}
            ]
            \input{data/tex/stats/balanced30-pool300.tex}
			\addlegendentry{Maximum}
			\addlegendentry{Median}
			\addlegendentry{Minimum}
		\end{axis}
	\end{tikzpicture}
	\caption{Detail of genetic algorithm on balanced data with 30 elements in set and pool size of 300}
\label{plot:genProfile30-pool300}
\end{figure}

			\addlegendentry{Maximum}
			\addlegendentry{Median}
			\addlegendentry{Minimum}
		\end{axis}
	\end{tikzpicture}
	\caption{Detail of genetic algorithm on balanced data with 30 elements in set and pool size of 300}
\label{plot:genProfile30-pool300}
\end{figure}

			\addlegendentry{Maximum}
			\addlegendentry{Median}
			\addlegendentry{Minimum}
		\end{axis}
	\end{tikzpicture}
	\caption{Detail of genetic algorithm on balanced data with 30 elements in set and pool size of 300}
\label{plot:genProfile30-pool300}
\end{figure}



\section{Variation in configuration variables}
When we take a look at \cref{plot:genProfile30-pool300} we can see the higher
the mutation coefficient is, the more spread out are our configurations, as
the chance for the breeding result to be completely random grows.

If we decide to use a larger pool for our algorithm, as you can see in \cref{plot:genProfile30-pool300},
where we used 300 elements, we gain the result in earlier generations. Moreover,
due to the nature of our fitness function, many of the medians have a fitness value of zero.

\section{Conclusion}
In conclusion, we have created and tested the genetic algorithm for the 0/1 version
of the knapsack problem. We have managed to calculate the result with the maximal
measured imprecision of 1\% with considerably low time complexity, especially for
greater instances of the knapsack problem.

\newpage
\bibliographystyle{iso690}
% \nocite{*} % all entries in the bib file
\bibliography{database.bib}

\end{document}