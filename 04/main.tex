
\NeedsTeXFormat{LaTeX2e}
\LoadClass{article}
\RequirePackage{todonotes}
\RequirePackage[parfill]{parskip}
\RequirePackage[margin=2.8cm]{geometry}
\RequirePackage{hyperref}
\RequirePackage[english]{babel}
\RequirePackage{pgfplots}
\RequirePackage{listings}
\RequirePackage{amsfonts}
\RequirePackage{subfiles}
\RequirePackage{mathtools}
\RequirePackage[noabbrev,capitalize,nameinlink]{cleveref}

\providecommand{\tightlist}{%
  \setlength{\itemsep}{0pt}\setlength{\parskip}{0pt}}

\begin{document}
\title{\textbf{MI-PAA~--~Task 4}\\
Genetic algorithm solving the optimisation version of the 0/1 knapsack problem}
\author{Daniel Hampl (hampldan)}
\date{\today}
\maketitle

\tableofcontents
\newpage

\section{Introduction}
The knapsack problem is one of the most widespread NP-Complete problems. It can be described as having a knapsack with a limited capacity and multiple items, where each of these items has a set value and weight. Our task is to fill the knapsack with items of highest combined value possible.

In this paper, we will be focussing approximation algorithms solving the knapsack problem and the deviation of results provided by these approximation algorithms.

\subsection{Definition\cite{WEBSITE:knapsackDef}}
We have weight $W$ and $n$ items, where item $i$ is described as a pair $(w_i, v_i)$.

\begin{itemize}
    \item $w_i$ is a weight of object $n_i$
    \item $v_i$ is a value of object $n_i$
    \item $w_i, v_i \in \mathbb{N}\setminus\{0\}$
\end{itemize}

Find value $V$, where:

\begin{itemize}
    \item $\sum_i(x_i*v_i) = V$
    \item $\sum_i(x_i*w_i) \leq W$
    \item $x_i \in \{0,1\} \forall i$
\end{itemize}

\subsection{Task}

Choose one heuristics (simulated annealing, genetic algorithm or tabu search)
and use it for solving the optimisation version of the 0/1 knapsack problem.
Test the solution on instances with at least 30 elements. Furthermore, test
your algorithm with different configuration variables.



\section{Implementation}
All of these algorithms are implemented in Python 3.7, and all data were gathered on the Windows10
OS running on Intel(R) Core(TM) i7-6700HQ CPU @ 2.60GHz.

In our genetic algorithm, we have used a fixed pool size for each to be 30, and the number of
generations has been set to ten times the size of the concrete instance.

\subsection{Fitness function}
For calculating the fitness of each configuration, we have used the price of this combination
with two additional conditions. The first condition being a case where the weight of said
combination is zero, and the second is the weight being higher, than the capacity of our knapsack.
In these cases, the value of our fitness function is set to 0 and -1, respectively.

\subsection{Selection of the next generation}
For selecting the next generation, we have decided to keep 2 elements from the last generation
as they are. As for the rest of the new generation, we have selected 10\% of the previous
generation and combined them with the last generation using random breeding.

The random breeding takes 2 combinations and randomly selects each bit from one of those,
which in turn creates a new combination used further in our algorithm.

\subsection{Mutation}
In every breeding, there is a chance of a mutation. During the selection of bits from parents,
there is a chance to flip the bit. This chance is 1 in 128; however, it significantly grows with
each cycle, where our fittest element stays unchanged.

Moreover, to ensure the potency of our population, we enforce uniqueness in our pool. To not
get stuck in an infinite loop, we also increase the chance of mutation with every step,
where we attempted to add an element, which was already in our pool.



\addplot[scatter,scatter src=explicit symbolic]table[meta=label] {
x y label
30 .024000 genetics
35 .022000 genetics
40 .025000 genetics
45 .030000 genetics
};
\addplot[scatter,scatter src=explicit symbolic]table[meta=label] {
x y label
30 1.666000 solveSmart
35 20.660000 solveSmart
40 118.983000 solveSmart
45 1492.431000 solveSmart
};


\addplot[scatter,scatter src=explicit symbolic]table[meta=label] {
x y label
30 .024000 genetics
35 .022000 genetics
40 .025000 genetics
45 .030000 genetics
};
\addplot[scatter,scatter src=explicit symbolic]table[meta=label] {
x y label
30 1.666000 solveSmart
35 20.660000 solveSmart
40 118.983000 solveSmart
45 1492.431000 solveSmart
};



\begin{figure}
	\centering
	\pgfplotsset{every axis legend/.append style={
		at={(1.05,0.5)},
		anchor=west}}
	\begin{tikzpicture}
		\begin{axis}[
			xlabel=Cycles passed,
			ylabel=Fitness,
			scatter/classes={
				max={mark=square*,blue},
				med={mark=square*,red},
				min={mark=square*,green}
				}
            ]
            \begin{figure}
	\centering
	\pgfplotsset{every axis legend/.append style={
		at={(1.05,0.5)},
		anchor=west}}
	\begin{tikzpicture}
		\begin{axis}[
			xlabel=Cycles passed,
			ylabel=Fitness,
			scatter/classes={
				max={mark=square*,blue},
				med={mark=square*,red},
				min={mark=square*,green}
				}
            ]
            \begin{figure}
	\centering
	\pgfplotsset{every axis legend/.append style={
		at={(1.05,0.5)},
		anchor=west}}
	\begin{tikzpicture}
		\begin{axis}[
			xlabel=Cycles passed,
			ylabel=Fitness,
			scatter/classes={
				max={mark=square*,blue},
				med={mark=square*,red},
				min={mark=square*,green}
				}
            ]
            \input{data/tex/stats/balanced30.tex}
			\addlegendentry{Maximum}
			\addlegendentry{Median}
			\addlegendentry{Minimum}
		\end{axis}
	\end{tikzpicture}
	\caption{Detail of genetic algorithm on balanced data with 30 elements in set}
\label{plot:genProfile30}
\end{figure}

			\addlegendentry{Maximum}
			\addlegendentry{Median}
			\addlegendentry{Minimum}
		\end{axis}
	\end{tikzpicture}
	\caption{Detail of genetic algorithm on balanced data with 30 elements in set}
\label{plot:genProfile30}
\end{figure}

			\addlegendentry{Maximum}
			\addlegendentry{Median}
			\addlegendentry{Minimum}
		\end{axis}
	\end{tikzpicture}
	\caption{Detail of genetic algorithm on balanced data with 30 elements in set}
\label{plot:genProfile30}
\end{figure}


\addplot+[scatter,only marks,scatter src=explicit symbolic]table[meta=label] {
x y label
0 354680 max
1 354680 max
2 354680 max
3 354680 max
4 354680 max
5 390664 max
6 390664 max
7 396776 max
8 396776 max
9 396776 max
10 396776 max
11 396776 max
12 403788 max
13 403788 max
14 403788 max
15 403788 max
16 412789 max
17 412789 max
18 412789 max
19 412789 max
20 412789 max
21 412789 max
22 412789 max
23 412789 max
24 412789 max
25 433180 max
26 433180 max
27 433180 max
28 433180 max
29 433180 max
30 433180 max
31 433180 max
32 433180 max
33 433180 max
34 433180 max
35 433180 max
36 433180 max
37 433180 max
38 433180 max
39 433180 max
40 435280 max
41 435280 max
42 435280 max
43 435280 max
44 435280 max
45 435280 max
46 435280 max
47 435280 max
48 435280 max
49 435280 max
50 435280 max
51 435280 max
52 435280 max
53 435280 max
54 435280 max
55 435280 max
56 435280 max
57 435280 max
58 435280 max
59 435280 max
60 435280 max
61 435280 max
62 435280 max
63 435280 max
64 443364 max
65 443364 max
66 443364 max
67 443364 max
68 443364 max
69 443364 max
70 443364 max
71 443364 max
72 443364 max
73 443364 max
74 443364 max
75 443364 max
76 443364 max
77 443364 max
78 443364 max
79 443364 max
80 443364 max
81 447418 max
82 447418 max
83 447418 max
84 447418 max
85 447418 max
86 447418 max
87 447418 max
88 447418 max
89 447418 max
90 447418 max
91 447418 max
92 447418 max
93 447418 max
94 447418 max
95 447418 max
96 447418 max
97 447418 max
98 447418 max
99 447418 max
100 447418 max
101 447418 max
102 447418 max
103 447418 max
104 447418 max
105 447418 max
106 447418 max
107 447418 max
108 447418 max
109 447418 max
110 447418 max
111 447418 max
112 447418 max
113 447418 max
114 447418 max
115 447418 max
116 447418 max
117 447418 max
118 447418 max
119 447418 max
120 447418 max
121 447418 max
122 447418 max
123 447418 max
124 447418 max
125 447418 max
126 447418 max
127 447418 max
128 447418 max
129 447418 max
130 447418 max
131 447418 max
132 447418 max
133 447418 max
134 447418 max
135 447418 max
136 447418 max
137 447418 max
138 447418 max
139 447418 max
140 447418 max
141 447418 max
142 447418 max
143 447418 max
144 447418 max
145 447418 max
146 447418 max
147 447418 max
148 447418 max
149 447418 max
150 447418 max
151 447418 max
152 447810 max
153 447810 max
154 447810 max
155 447810 max
156 447810 max
157 447810 max
158 447810 max
159 447810 max
160 447810 max
161 447810 max
162 447810 max
163 447810 max
164 447810 max
165 448617 max
166 448617 max
167 448617 max
168 448617 max
169 448617 max
170 448617 max
171 448617 max
172 448617 max
173 448617 max
174 448617 max
175 448617 max
176 448617 max
177 448617 max
178 448617 max
179 448617 max
180 448617 max
181 448617 max
182 448617 max
183 448617 max
184 448617 max
185 448617 max
186 448617 max
187 448617 max
188 448617 max
189 448617 max
190 448617 max
191 448617 max
192 448617 max
193 448617 max
194 448617 max
195 448617 max
196 448617 max
197 449395 max
198 449395 max
199 449395 max
200 449395 max
201 449395 max
202 449395 max
203 449395 max
204 449395 max
205 449395 max
206 449395 max
207 449395 max
208 449395 max
209 449395 max
210 449395 max
211 449395 max
212 449395 max
213 449395 max
214 449442 max
215 449442 max
216 449442 max
217 450547 max
218 450547 max
219 450547 max
220 450547 max
221 450547 max
222 450547 max
223 450547 max
224 450547 max
225 450594 max
226 450594 max
227 450594 max
228 450594 max
229 450594 max
230 450594 max
231 450594 max
232 450594 max
233 450594 max
234 450594 max
235 450594 max
236 450594 max
237 450594 max
238 450594 max
239 450594 max
240 450594 max
241 450594 max
242 450594 max
243 450594 max
244 450594 max
245 450594 max
246 450594 max
247 450594 max
248 450594 max
249 455224 max
250 455224 max
251 455224 max
252 455224 max
253 455224 max
254 455224 max
255 455224 max
256 455224 max
257 455224 max
258 455224 max
259 455224 max
260 455224 max
261 455224 max
262 455224 max
263 455224 max
264 455224 max
265 455224 max
266 455224 max
267 455224 max
268 455224 max
269 455224 max
270 455224 max
271 455224 max
272 455224 max
273 455224 max
274 455224 max
275 455224 max
276 455224 max
277 455224 max
278 455224 max
279 455224 max
280 455224 max
281 455224 max
282 455224 max
283 455224 max
284 455224 max
285 455224 max
286 455224 max
287 455224 max
288 455224 max
289 455224 max
290 455224 max
291 455224 max
292 455224 max
293 455224 max
294 455224 max
295 455224 max
296 455224 max
297 455224 max
298 455224 max
299 455224 max
300 455224 max
301 455224 max
302 455224 max
303 455224 max
304 455224 max
305 455224 max
306 455224 max
307 455224 max
308 455224 max
309 455224 max
310 455224 max
311 455224 max
312 455224 max
313 455224 max
314 455224 max
315 455224 max
316 455224 max
317 455224 max
318 455224 max
319 455224 max
320 455224 max
321 455224 max
322 455224 max
323 455224 max
324 455224 max
325 455224 max
326 455224 max
327 455224 max
328 455224 max
329 455224 max
330 455224 max
331 455224 max
332 455224 max
333 455224 max
334 455224 max
335 455224 max
336 455224 max
337 455224 max
338 455224 max
339 455224 max
340 455224 max
341 455224 max
342 455224 max
343 455224 max
344 455224 max
345 455224 max
346 455224 max
347 455224 max
348 455224 max
349 455224 max
};
\addplot+[scatter,only marks,scatter src=explicit symbolic]table[meta=label] {
x y label
0 0 med
1 211699 med
2 257852 med
3 288415 med
4 285891 med
5 285040 med
6 307498 med
7 303793 med
8 302051 med
9 336141 med
10 265379 med
11 288994 med
12 293017 med
13 270079 med
14 318206 med
15 314122 med
16 302052 med
17 289715 med
18 0 med
19 0 med
20 328222 med
21 282179 med
22 256481 med
23 323417 med
24 0 med
25 254478 med
26 328543 med
27 327296 med
28 326165 med
29 329800 med
30 291460 med
31 284028 med
32 318130 med
33 322482 med
34 266980 med
35 309091 med
36 305234 med
37 312794 med
38 265272 med
39 316200 med
40 246915 med
41 259286 med
42 360626 med
43 354607 med
44 308660 med
45 329607 med
46 326342 med
47 317386 med
48 0 med
49 313123 med
50 268850 med
51 319782 med
52 336937 med
53 352685 med
54 329761 med
55 312868 med
56 311359 med
57 324762 med
58 0 med
59 260191 med
60 322595 med
61 317170 med
62 335027 med
63 0 med
64 323896 med
65 292589 med
66 321958 med
67 336464 med
68 303504 med
69 325658 med
70 332248 med
71 349668 med
72 335874 med
73 324335 med
74 351481 med
75 300578 med
76 0 med
77 0 med
78 289155 med
79 322986 med
80 0 med
81 0 med
82 311244 med
83 272157 med
84 272882 med
85 311652 med
86 324178 med
87 351385 med
88 332683 med
89 309591 med
90 330401 med
91 273772 med
92 313019 med
93 353679 med
94 331329 med
95 342899 med
96 371871 med
97 344792 med
98 380465 med
99 286071 med
100 354641 med
101 319144 med
102 0 med
103 362110 med
104 315921 med
105 283051 med
106 269878 med
107 332750 med
108 350724 med
109 309680 med
110 333235 med
111 328404 med
112 318873 med
113 326750 med
114 335405 med
115 296498 med
116 359041 med
117 322553 med
118 335187 med
119 318712 med
120 280281 med
121 0 med
122 0 med
123 314834 med
124 322474 med
125 345724 med
126 327763 med
127 331264 med
128 346662 med
129 358321 med
130 350233 med
131 325400 med
132 352242 med
133 0 med
134 315006 med
135 317810 med
136 250742 med
137 273011 med
138 285470 med
139 318549 med
140 336896 med
141 347617 med
142 348453 med
143 285091 med
144 346802 med
145 336207 med
146 330375 med
147 341497 med
148 0 med
149 301333 med
150 327356 med
151 318154 med
152 290258 med
153 329508 med
154 298411 med
155 320923 med
156 321093 med
157 283076 med
158 314129 med
159 324667 med
160 287125 med
161 354489 med
162 338838 med
163 273634 med
164 339667 med
165 309217 med
166 354392 med
167 322276 med
168 343332 med
169 259276 med
170 307100 med
171 367416 med
172 323161 med
173 343336 med
174 305503 med
175 283488 med
176 306929 med
177 343799 med
178 327021 med
179 314049 med
180 322325 med
181 306859 med
182 313699 med
183 324145 med
184 314121 med
185 325329 med
186 333240 med
187 294788 med
188 0 med
189 0 med
190 295847 med
191 330732 med
192 0 med
193 253140 med
194 331421 med
195 0 med
196 352635 med
197 329153 med
198 0 med
199 0 med
200 347300 med
201 337142 med
202 351634 med
203 341897 med
204 0 med
205 328496 med
206 351957 med
207 313260 med
208 336930 med
209 332540 med
210 337909 med
211 287113 med
212 280179 med
213 323917 med
214 357643 med
215 320949 med
216 355526 med
217 325258 med
218 315844 med
219 294106 med
220 322413 med
221 317030 med
222 350275 med
223 353321 med
224 362071 med
225 327594 med
226 300911 med
227 332889 med
228 296632 med
229 0 med
230 0 med
231 322874 med
232 296086 med
233 263362 med
234 333811 med
235 326915 med
236 304221 med
237 264371 med
238 366900 med
239 316148 med
240 265939 med
241 344436 med
242 308785 med
243 0 med
244 0 med
245 313303 med
246 274649 med
247 333320 med
248 343317 med
249 301070 med
250 237349 med
251 306156 med
252 263579 med
253 308620 med
254 0 med
255 344183 med
256 346115 med
257 325982 med
258 313155 med
259 319626 med
260 0 med
261 355008 med
262 0 med
263 291991 med
264 326758 med
265 317182 med
266 273238 med
267 359896 med
268 346850 med
269 365035 med
270 319860 med
271 328719 med
272 337036 med
273 313107 med
274 301233 med
275 350291 med
276 341462 med
277 322418 med
278 336469 med
279 0 med
280 0 med
281 307469 med
282 363612 med
283 293661 med
284 284985 med
285 349338 med
286 353902 med
287 288116 med
288 316666 med
289 323950 med
290 296233 med
291 0 med
292 313726 med
293 321381 med
294 332609 med
295 255213 med
296 285968 med
297 220965 med
298 304162 med
299 313977 med
300 304184 med
301 315584 med
302 0 med
303 330535 med
304 337051 med
305 343454 med
306 330004 med
307 307142 med
308 237350 med
309 288841 med
310 347588 med
311 301712 med
312 329757 med
313 0 med
314 0 med
315 322675 med
316 363359 med
317 321314 med
318 352183 med
319 283717 med
320 0 med
321 0 med
322 0 med
323 283704 med
324 338074 med
325 321515 med
326 0 med
327 0 med
328 306148 med
329 0 med
330 328453 med
331 312668 med
332 308933 med
333 300194 med
334 326709 med
335 317408 med
336 351398 med
337 335545 med
338 298950 med
339 328767 med
340 343406 med
341 303460 med
342 347043 med
343 268501 med
344 268011 med
345 284296 med
346 315097 med
347 352591 med
348 331572 med
349 360171 med
};
\addplot+[scatter,only marks,scatter src=explicit symbolic]table[meta=label] {
x y label
0 0 min
1 0 min
2 0 min
3 0 min
4 0 min
5 0 min
6 0 min
7 0 min
8 0 min
9 0 min
10 0 min
11 0 min
12 0 min
13 0 min
14 0 min
15 0 min
16 0 min
17 0 min
18 0 min
19 0 min
20 0 min
21 0 min
22 0 min
23 0 min
24 0 min
25 0 min
26 0 min
27 0 min
28 0 min
29 0 min
30 0 min
31 0 min
32 0 min
33 0 min
34 0 min
35 0 min
36 0 min
37 0 min
38 0 min
39 0 min
40 0 min
41 0 min
42 0 min
43 0 min
44 0 min
45 0 min
46 0 min
47 0 min
48 0 min
49 0 min
50 0 min
51 0 min
52 0 min
53 0 min
54 0 min
55 0 min
56 0 min
57 0 min
58 0 min
59 0 min
60 0 min
61 0 min
62 0 min
63 0 min
64 0 min
65 0 min
66 0 min
67 0 min
68 0 min
69 0 min
70 0 min
71 0 min
72 0 min
73 0 min
74 0 min
75 0 min
76 0 min
77 0 min
78 0 min
79 0 min
80 0 min
81 0 min
82 0 min
83 0 min
84 0 min
85 0 min
86 0 min
87 0 min
88 0 min
89 0 min
90 0 min
91 0 min
92 0 min
93 0 min
94 0 min
95 0 min
96 0 min
97 0 min
98 0 min
99 0 min
100 0 min
101 0 min
102 0 min
103 0 min
104 0 min
105 0 min
106 0 min
107 0 min
108 0 min
109 0 min
110 0 min
111 0 min
112 0 min
113 0 min
114 0 min
115 0 min
116 0 min
117 0 min
118 0 min
119 0 min
120 0 min
121 0 min
122 0 min
123 0 min
124 0 min
125 0 min
126 0 min
127 0 min
128 0 min
129 0 min
130 0 min
131 0 min
132 0 min
133 0 min
134 0 min
135 0 min
136 0 min
137 0 min
138 0 min
139 0 min
140 0 min
141 0 min
142 0 min
143 0 min
144 0 min
145 0 min
146 0 min
147 0 min
148 0 min
149 0 min
150 0 min
151 0 min
152 0 min
153 0 min
154 0 min
155 0 min
156 0 min
157 0 min
158 0 min
159 0 min
160 0 min
161 0 min
162 0 min
163 0 min
164 0 min
165 0 min
166 0 min
167 0 min
168 0 min
169 0 min
170 0 min
171 0 min
172 0 min
173 0 min
174 0 min
175 0 min
176 0 min
177 0 min
178 0 min
179 0 min
180 0 min
181 0 min
182 0 min
183 0 min
184 0 min
185 0 min
186 0 min
187 0 min
188 0 min
189 0 min
190 0 min
191 0 min
192 0 min
193 0 min
194 0 min
195 0 min
196 0 min
197 0 min
198 0 min
199 0 min
200 0 min
201 0 min
202 0 min
203 0 min
204 0 min
205 0 min
206 0 min
207 0 min
208 0 min
209 0 min
210 0 min
211 0 min
212 0 min
213 0 min
214 0 min
215 0 min
216 0 min
217 0 min
218 0 min
219 0 min
220 0 min
221 0 min
222 0 min
223 0 min
224 0 min
225 0 min
226 0 min
227 0 min
228 0 min
229 0 min
230 0 min
231 0 min
232 0 min
233 0 min
234 0 min
235 0 min
236 0 min
237 0 min
238 0 min
239 0 min
240 0 min
241 0 min
242 0 min
243 0 min
244 0 min
245 0 min
246 0 min
247 0 min
248 0 min
249 0 min
250 0 min
251 0 min
252 0 min
253 0 min
254 0 min
255 0 min
256 0 min
257 0 min
258 0 min
259 0 min
260 0 min
261 0 min
262 0 min
263 0 min
264 0 min
265 0 min
266 0 min
267 0 min
268 0 min
269 0 min
270 0 min
271 0 min
272 0 min
273 0 min
274 0 min
275 0 min
276 0 min
277 0 min
278 0 min
279 0 min
280 0 min
281 0 min
282 0 min
283 0 min
284 0 min
285 0 min
286 0 min
287 0 min
288 0 min
289 0 min
290 0 min
291 0 min
292 0 min
293 0 min
294 0 min
295 0 min
296 0 min
297 0 min
298 0 min
299 0 min
300 0 min
301 0 min
302 0 min
303 0 min
304 0 min
305 0 min
306 0 min
307 0 min
308 0 min
309 0 min
310 0 min
311 0 min
312 0 min
313 0 min
314 0 min
315 0 min
316 0 min
317 0 min
318 0 min
319 0 min
320 0 min
321 0 min
322 0 min
323 0 min
324 0 min
325 0 min
326 0 min
327 0 min
328 0 min
329 0 min
330 0 min
331 0 min
332 0 min
333 0 min
334 0 min
335 0 min
336 0 min
337 0 min
338 0 min
339 0 min
340 0 min
341 0 min
342 0 min
343 0 min
344 0 min
345 0 min
346 0 min
347 0 min
348 0 min
349 0 min
};


\addplot+[scatter,only marks,scatter src=explicit symbolic]table[meta=label] {
x y label
0 418318 max
1 418318 max
2 451042 max
3 451042 max
4 451042 max
5 451042 max
6 451042 max
7 451042 max
8 451042 max
9 451042 max
10 451042 max
11 451042 max
12 451042 max
13 451042 max
14 474773 max
15 474773 max
16 474773 max
17 474773 max
18 479585 max
19 482122 max
20 484499 max
21 484499 max
22 484499 max
23 484499 max
24 484499 max
25 484499 max
26 484499 max
27 484499 max
28 494226 max
29 512351 max
30 512351 max
31 512351 max
32 512351 max
33 512351 max
34 512351 max
35 512351 max
36 512351 max
37 512351 max
38 512351 max
39 512351 max
40 512351 max
41 512351 max
42 512351 max
43 512351 max
44 512351 max
45 512351 max
46 512351 max
47 512351 max
48 512351 max
49 512351 max
50 512351 max
51 512351 max
52 512351 max
53 512351 max
54 512351 max
55 512351 max
56 512351 max
57 512351 max
58 512351 max
59 512351 max
60 512351 max
61 512351 max
62 512351 max
63 512351 max
64 512351 max
65 512351 max
66 512351 max
67 512351 max
68 512351 max
69 512351 max
70 512351 max
71 512351 max
72 512351 max
73 512351 max
74 512351 max
75 512351 max
76 512351 max
77 512351 max
78 512351 max
79 512351 max
80 512351 max
81 512351 max
82 512351 max
83 512351 max
84 512351 max
85 512351 max
86 512351 max
87 512351 max
88 512351 max
89 512351 max
90 512351 max
91 512351 max
92 512351 max
93 512351 max
94 512351 max
95 512351 max
96 512351 max
97 512351 max
98 512351 max
99 512351 max
100 512351 max
101 512351 max
102 512351 max
103 512351 max
104 512351 max
105 512351 max
106 512351 max
107 512351 max
108 512351 max
109 512351 max
110 512351 max
111 512351 max
112 512351 max
113 512351 max
114 512351 max
115 512351 max
116 512351 max
117 512351 max
118 512351 max
119 512351 max
120 512351 max
121 512351 max
122 512351 max
123 512351 max
124 512351 max
125 512351 max
126 512351 max
127 512351 max
128 512351 max
129 512351 max
130 512351 max
131 512351 max
132 512351 max
133 512351 max
134 512351 max
135 512351 max
136 512351 max
137 512351 max
138 512351 max
139 512351 max
140 512351 max
141 512351 max
142 512351 max
143 512351 max
144 512351 max
145 512351 max
146 517373 max
147 517373 max
148 517373 max
149 517373 max
150 517373 max
151 517373 max
152 517373 max
153 517373 max
154 517373 max
155 517373 max
156 517373 max
157 517373 max
158 517373 max
159 517373 max
160 517373 max
161 517373 max
162 517373 max
163 517373 max
164 517373 max
165 517373 max
166 517373 max
167 517373 max
168 517373 max
169 517373 max
170 517373 max
171 517373 max
172 517373 max
173 517373 max
174 517373 max
175 517373 max
176 517373 max
177 517373 max
178 517373 max
179 517373 max
180 517373 max
181 517373 max
182 517373 max
183 517373 max
184 517373 max
185 517373 max
186 517373 max
187 517373 max
188 517373 max
189 517373 max
190 517373 max
191 517373 max
192 517373 max
193 517373 max
194 517373 max
195 517373 max
196 517373 max
197 517373 max
198 517373 max
199 517373 max
200 517373 max
201 517373 max
202 517373 max
203 517373 max
204 517373 max
205 517373 max
206 517373 max
207 517373 max
208 517373 max
209 517373 max
210 517373 max
211 517373 max
212 517373 max
213 517373 max
214 517373 max
215 517373 max
216 517373 max
217 517373 max
218 517373 max
219 517373 max
220 517373 max
221 517373 max
222 517373 max
223 517373 max
224 517373 max
225 517373 max
226 517373 max
227 517373 max
228 517373 max
229 517373 max
230 517373 max
231 517373 max
232 517373 max
233 517373 max
234 517373 max
235 517373 max
236 517373 max
237 517373 max
238 517373 max
239 517373 max
240 517373 max
241 517373 max
242 517373 max
243 517373 max
244 517373 max
245 517373 max
246 517373 max
247 517373 max
248 517373 max
249 517373 max
250 517373 max
251 517373 max
252 517373 max
253 517373 max
254 517373 max
255 517373 max
256 517373 max
257 517373 max
258 517373 max
259 517373 max
260 517373 max
261 519450 max
262 519450 max
263 519450 max
264 519450 max
265 519450 max
266 519450 max
267 519450 max
268 519450 max
269 519450 max
270 519450 max
271 519450 max
272 519450 max
273 519450 max
274 519450 max
275 519450 max
276 519450 max
277 519450 max
278 519450 max
279 519450 max
280 519450 max
281 519450 max
282 519450 max
283 519450 max
284 519450 max
285 519450 max
286 519450 max
287 519450 max
288 519450 max
289 519450 max
290 519450 max
291 519450 max
292 519450 max
293 519450 max
294 519450 max
295 519450 max
296 519450 max
297 519450 max
298 519450 max
299 519450 max
300 519450 max
301 519450 max
302 519450 max
303 519450 max
304 519450 max
305 519450 max
306 519450 max
307 519450 max
308 519450 max
309 519450 max
310 519450 max
311 519450 max
312 519450 max
313 519450 max
314 519450 max
315 519450 max
316 519450 max
317 519450 max
318 519450 max
319 519450 max
320 519450 max
321 519450 max
322 519450 max
323 519450 max
324 519450 max
325 519450 max
326 519450 max
327 519450 max
328 519450 max
329 519450 max
330 519450 max
331 519450 max
332 519450 max
333 519450 max
334 519450 max
335 519450 max
336 519450 max
337 519450 max
338 519450 max
339 519450 max
340 519450 max
341 519450 max
342 519450 max
343 519450 max
344 519450 max
345 519450 max
346 519450 max
347 519450 max
348 519450 max
349 519450 max
350 519450 max
351 519450 max
352 519450 max
353 519450 max
354 519450 max
355 519450 max
356 519450 max
357 519450 max
358 519450 max
359 519450 max
360 519450 max
361 519450 max
362 519450 max
363 519450 max
364 519450 max
365 519450 max
366 519450 max
367 519450 max
368 519450 max
369 519450 max
370 519450 max
371 519450 max
372 519450 max
373 519450 max
374 519450 max
375 519450 max
376 519450 max
377 519450 max
378 519450 max
379 519450 max
380 519450 max
381 519450 max
382 519450 max
383 519450 max
384 519450 max
385 519450 max
386 519450 max
387 519450 max
388 519450 max
389 519450 max
390 519450 max
391 519450 max
392 519450 max
393 519450 max
394 519450 max
395 519450 max
396 519450 max
397 519450 max
398 519450 max
399 519450 max
};
\addplot+[scatter,only marks,scatter src=explicit symbolic]table[meta=label] {
x y label
0 239097 med
1 235804 med
2 272837 med
3 310884 med
4 316151 med
5 347185 med
6 350081 med
7 365861 med
8 353244 med
9 319627 med
10 361191 med
11 335571 med
12 302248 med
13 387975 med
14 325963 med
15 296754 med
16 372259 med
17 363654 med
18 351233 med
19 313382 med
20 332682 med
21 281230 med
22 328743 med
23 345530 med
24 359712 med
25 0 med
26 338517 med
27 0 med
28 328891 med
29 360891 med
30 403797 med
31 389605 med
32 401260 med
33 384204 med
34 371640 med
35 340875 med
36 374785 med
37 389827 med
38 0 med
39 316401 med
40 355133 med
41 384404 med
42 393869 med
43 346331 med
44 350473 med
45 391619 med
46 352524 med
47 377196 med
48 393003 med
49 0 med
50 0 med
51 391440 med
52 288583 med
53 353795 med
54 350133 med
55 403285 med
56 401673 med
57 382316 med
58 392689 med
59 386390 med
60 366043 med
61 412730 med
62 390147 med
63 361884 med
64 341127 med
65 373793 med
66 391254 med
67 377359 med
68 0 med
69 374946 med
70 401803 med
71 384045 med
72 318587 med
73 411533 med
74 395604 med
75 0 med
76 357913 med
77 0 med
78 399499 med
79 351571 med
80 393217 med
81 362200 med
82 391426 med
83 364099 med
84 392842 med
85 352639 med
86 0 med
87 391942 med
88 401616 med
89 412397 med
90 391665 med
91 342763 med
92 375271 med
93 395033 med
94 346264 med
95 352404 med
96 0 med
97 0 med
98 409638 med
99 336807 med
100 0 med
101 0 med
102 313974 med
103 372506 med
104 384130 med
105 362635 med
106 373063 med
107 322703 med
108 0 med
109 394752 med
110 326958 med
111 332397 med
112 318810 med
113 379032 med
114 349822 med
115 367760 med
116 396272 med
117 0 med
118 358075 med
119 315454 med
120 325049 med
121 381639 med
122 0 med
123 0 med
124 0 med
125 0 med
126 0 med
127 350540 med
128 375577 med
129 386202 med
130 400979 med
131 339544 med
132 333335 med
133 378233 med
134 338876 med
135 0 med
136 356638 med
137 307273 med
138 0 med
139 377396 med
140 374142 med
141 402481 med
142 337523 med
143 397392 med
144 365039 med
145 352052 med
146 389734 med
147 402084 med
148 342394 med
149 348439 med
150 385394 med
151 350538 med
152 366001 med
153 0 med
154 0 med
155 0 med
156 362453 med
157 344586 med
158 390425 med
159 0 med
160 323165 med
161 0 med
162 303852 med
163 361917 med
164 391875 med
165 342514 med
166 377579 med
167 383022 med
168 0 med
169 374626 med
170 357000 med
171 367218 med
172 377048 med
173 0 med
174 371804 med
175 376787 med
176 396568 med
177 405235 med
178 0 med
179 342172 med
180 370701 med
181 373379 med
182 389720 med
183 357320 med
184 358121 med
185 321821 med
186 364536 med
187 335589 med
188 388281 med
189 0 med
190 0 med
191 0 med
192 335058 med
193 362392 med
194 375992 med
195 339274 med
196 273306 med
197 0 med
198 361038 med
199 410431 med
200 0 med
201 350445 med
202 304482 med
203 368835 med
204 387799 med
205 348973 med
206 401903 med
207 420928 med
208 380555 med
209 344708 med
210 364363 med
211 0 med
212 378922 med
213 0 med
214 390233 med
215 376339 med
216 0 med
217 0 med
218 0 med
219 404008 med
220 320053 med
221 410627 med
222 408673 med
223 0 med
224 394778 med
225 391092 med
226 366368 med
227 406512 med
228 397300 med
229 347129 med
230 0 med
231 354964 med
232 393523 med
233 410073 med
234 0 med
235 390309 med
236 350361 med
237 309961 med
238 393095 med
239 399924 med
240 398131 med
241 0 med
242 314039 med
243 337516 med
244 0 med
245 0 med
246 0 med
247 0 med
248 316903 med
249 364331 med
250 365224 med
251 377391 med
252 416421 med
253 338463 med
254 362842 med
255 362961 med
256 392162 med
257 315516 med
258 403949 med
259 395876 med
260 360441 med
261 0 med
262 0 med
263 0 med
264 373100 med
265 377649 med
266 0 med
267 369165 med
268 368131 med
269 386559 med
270 341799 med
271 324990 med
272 343750 med
273 366387 med
274 396148 med
275 394276 med
276 398727 med
277 322469 med
278 326108 med
279 377306 med
280 403219 med
281 344909 med
282 375542 med
283 365574 med
284 395435 med
285 398483 med
286 380499 med
287 266437 med
288 365712 med
289 389148 med
290 381684 med
291 0 med
292 373083 med
293 384401 med
294 363994 med
295 338158 med
296 0 med
297 364186 med
298 352087 med
299 388736 med
300 389877 med
301 399442 med
302 377495 med
303 379319 med
304 408508 med
305 376317 med
306 351158 med
307 375312 med
308 343474 med
309 382874 med
310 407600 med
311 422579 med
312 397273 med
313 395194 med
314 365173 med
315 382621 med
316 406477 med
317 0 med
318 355613 med
319 385083 med
320 337564 med
321 356860 med
322 382640 med
323 0 med
324 0 med
325 376349 med
326 319924 med
327 299036 med
328 368831 med
329 376390 med
330 396832 med
331 366066 med
332 355378 med
333 297034 med
334 387064 med
335 309878 med
336 350467 med
337 366400 med
338 383987 med
339 340624 med
340 371195 med
341 352418 med
342 383965 med
343 0 med
344 377691 med
345 429821 med
346 407172 med
347 369955 med
348 0 med
349 0 med
350 310443 med
351 0 med
352 332904 med
353 0 med
354 308117 med
355 394013 med
356 389794 med
357 0 med
358 386976 med
359 383120 med
360 377110 med
361 378705 med
362 350039 med
363 409184 med
364 271618 med
365 382001 med
366 331341 med
367 319466 med
368 0 med
369 376011 med
370 0 med
371 371905 med
372 353739 med
373 352339 med
374 386492 med
375 0 med
376 369547 med
377 298598 med
378 401855 med
379 386534 med
380 0 med
381 349314 med
382 296181 med
383 399261 med
384 392325 med
385 402528 med
386 0 med
387 353130 med
388 403752 med
389 372736 med
390 393380 med
391 276695 med
392 399627 med
393 407071 med
394 381431 med
395 275823 med
396 380952 med
397 0 med
398 0 med
399 0 med
};
\addplot+[scatter,only marks,scatter src=explicit symbolic]table[meta=label] {
x y label
0 0 min
1 0 min
2 0 min
3 0 min
4 0 min
5 0 min
6 0 min
7 0 min
8 0 min
9 0 min
10 0 min
11 0 min
12 0 min
13 0 min
14 0 min
15 0 min
16 0 min
17 0 min
18 0 min
19 0 min
20 0 min
21 0 min
22 0 min
23 0 min
24 0 min
25 0 min
26 0 min
27 0 min
28 0 min
29 0 min
30 0 min
31 0 min
32 0 min
33 0 min
34 0 min
35 0 min
36 0 min
37 0 min
38 0 min
39 0 min
40 0 min
41 0 min
42 0 min
43 0 min
44 0 min
45 0 min
46 0 min
47 0 min
48 0 min
49 0 min
50 0 min
51 0 min
52 0 min
53 0 min
54 0 min
55 0 min
56 0 min
57 0 min
58 0 min
59 0 min
60 0 min
61 0 min
62 0 min
63 0 min
64 0 min
65 0 min
66 0 min
67 0 min
68 0 min
69 0 min
70 0 min
71 0 min
72 0 min
73 0 min
74 0 min
75 0 min
76 0 min
77 0 min
78 0 min
79 0 min
80 0 min
81 0 min
82 0 min
83 0 min
84 0 min
85 0 min
86 0 min
87 0 min
88 0 min
89 0 min
90 0 min
91 0 min
92 0 min
93 0 min
94 0 min
95 0 min
96 0 min
97 0 min
98 0 min
99 0 min
100 0 min
101 0 min
102 0 min
103 0 min
104 0 min
105 0 min
106 0 min
107 0 min
108 0 min
109 0 min
110 0 min
111 0 min
112 0 min
113 0 min
114 0 min
115 0 min
116 0 min
117 0 min
118 0 min
119 0 min
120 0 min
121 0 min
122 0 min
123 0 min
124 0 min
125 0 min
126 0 min
127 0 min
128 0 min
129 0 min
130 0 min
131 0 min
132 0 min
133 0 min
134 0 min
135 0 min
136 0 min
137 0 min
138 0 min
139 0 min
140 0 min
141 0 min
142 0 min
143 0 min
144 0 min
145 0 min
146 0 min
147 0 min
148 0 min
149 0 min
150 0 min
151 0 min
152 0 min
153 0 min
154 0 min
155 0 min
156 0 min
157 0 min
158 0 min
159 0 min
160 0 min
161 0 min
162 0 min
163 0 min
164 0 min
165 0 min
166 0 min
167 0 min
168 0 min
169 0 min
170 0 min
171 0 min
172 0 min
173 0 min
174 0 min
175 0 min
176 0 min
177 0 min
178 0 min
179 0 min
180 0 min
181 0 min
182 0 min
183 0 min
184 0 min
185 0 min
186 0 min
187 0 min
188 0 min
189 0 min
190 0 min
191 0 min
192 0 min
193 0 min
194 0 min
195 0 min
196 0 min
197 0 min
198 0 min
199 0 min
200 0 min
201 0 min
202 0 min
203 0 min
204 0 min
205 0 min
206 0 min
207 0 min
208 0 min
209 0 min
210 0 min
211 0 min
212 0 min
213 0 min
214 0 min
215 0 min
216 0 min
217 0 min
218 0 min
219 0 min
220 0 min
221 0 min
222 0 min
223 0 min
224 0 min
225 0 min
226 0 min
227 0 min
228 0 min
229 0 min
230 0 min
231 0 min
232 0 min
233 0 min
234 0 min
235 0 min
236 0 min
237 0 min
238 0 min
239 0 min
240 0 min
241 0 min
242 0 min
243 0 min
244 0 min
245 0 min
246 0 min
247 0 min
248 0 min
249 0 min
250 0 min
251 0 min
252 0 min
253 0 min
254 0 min
255 0 min
256 0 min
257 0 min
258 0 min
259 0 min
260 0 min
261 0 min
262 0 min
263 0 min
264 0 min
265 0 min
266 0 min
267 0 min
268 0 min
269 0 min
270 0 min
271 0 min
272 0 min
273 0 min
274 0 min
275 0 min
276 0 min
277 0 min
278 0 min
279 0 min
280 0 min
281 0 min
282 0 min
283 0 min
284 0 min
285 0 min
286 0 min
287 0 min
288 0 min
289 0 min
290 0 min
291 0 min
292 0 min
293 0 min
294 0 min
295 0 min
296 0 min
297 0 min
298 0 min
299 0 min
300 0 min
301 0 min
302 0 min
303 0 min
304 0 min
305 0 min
306 0 min
307 0 min
308 0 min
309 0 min
310 0 min
311 0 min
312 0 min
313 0 min
314 0 min
315 0 min
316 0 min
317 0 min
318 0 min
319 0 min
320 0 min
321 0 min
322 0 min
323 0 min
324 0 min
325 0 min
326 0 min
327 0 min
328 0 min
329 0 min
330 0 min
331 0 min
332 0 min
333 0 min
334 0 min
335 0 min
336 0 min
337 0 min
338 0 min
339 0 min
340 0 min
341 0 min
342 0 min
343 0 min
344 0 min
345 0 min
346 0 min
347 0 min
348 0 min
349 0 min
350 0 min
351 0 min
352 0 min
353 0 min
354 0 min
355 0 min
356 0 min
357 0 min
358 0 min
359 0 min
360 0 min
361 0 min
362 0 min
363 0 min
364 0 min
365 0 min
366 0 min
367 0 min
368 0 min
369 0 min
370 0 min
371 0 min
372 0 min
373 0 min
374 0 min
375 0 min
376 0 min
377 0 min
378 0 min
379 0 min
380 0 min
381 0 min
382 0 min
383 0 min
384 0 min
385 0 min
386 0 min
387 0 min
388 0 min
389 0 min
390 0 min
391 0 min
392 0 min
393 0 min
394 0 min
395 0 min
396 0 min
397 0 min
398 0 min
399 0 min
};

\begin{figure}
	\centering
	\pgfplotsset{every axis legend/.append style={
		at={(1.05,0.5)},
		anchor=west}}
	\begin{tikzpicture}
		\begin{axis}[
			xlabel=Cycles passed,
			ylabel=Fitness,
			scatter/classes={
				max={mark=square*,blue},
				med={mark=square*,red},
				min={mark=square*,green}
				}
            ]
            \begin{figure}
	\centering
	\pgfplotsset{every axis legend/.append style={
		at={(1.05,0.5)},
		anchor=west}}
	\begin{tikzpicture}
		\begin{axis}[
			xlabel=Cycles passed,
			ylabel=Fitness,
			scatter/classes={
				max={mark=square*,blue},
				med={mark=square*,red},
				min={mark=square*,green}
				}
            ]
            \begin{figure}
	\centering
	\pgfplotsset{every axis legend/.append style={
		at={(1.05,0.5)},
		anchor=west}}
	\begin{tikzpicture}
		\begin{axis}[
			xlabel=Cycles passed,
			ylabel=Fitness,
			scatter/classes={
				max={mark=square*,blue},
				med={mark=square*,red},
				min={mark=square*,green}
				}
            ]
            \input{data/tex/stats/balanced45.tex}
			\addlegendentry{Maximum}
			\addlegendentry{Median}
			\addlegendentry{Minimum}
		\end{axis}
	\end{tikzpicture}
	\caption{Detail of genetic algorithm on balanced data with 45 elements in set}
\label{plot:genProfile45}
\end{figure}

			\addlegendentry{Maximum}
			\addlegendentry{Median}
			\addlegendentry{Minimum}
		\end{axis}
	\end{tikzpicture}
	\caption{Detail of genetic algorithm on balanced data with 45 elements in set}
\label{plot:genProfile45}
\end{figure}

			\addlegendentry{Maximum}
			\addlegendentry{Median}
			\addlegendentry{Minimum}
		\end{axis}
	\end{tikzpicture}
	\caption{Detail of genetic algorithm on balanced data with 45 elements in set}
\label{plot:genProfile45}
\end{figure}




\addplot+[scatter,only marks,scatter src=explicit symbolic]table[meta=label] {
x y label
0 272962 max
1 294172 max
2 304979 max
3 304979 max
4 322377 max
5 325684 max
6 344312 max
7 355718 max
8 355718 max
9 355718 max
10 365209 max
11 365209 max
12 365209 max
13 365209 max
14 372043 max
15 372043 max
16 372043 max
17 372043 max
18 372043 max
19 372043 max
20 375070 max
21 376222 max
22 376222 max
23 376269 max
24 376269 max
25 376269 max
26 376269 max
27 376269 max
28 376269 max
29 376269 max
30 376269 max
31 376269 max
32 376269 max
33 376269 max
34 376269 max
35 376269 max
36 376269 max
37 376269 max
38 376269 max
39 376269 max
40 376269 max
41 376269 max
42 376269 max
43 376269 max
44 376269 max
45 376269 max
46 376269 max
47 376269 max
48 379206 max
49 379206 max
50 379206 max
51 379206 max
52 379206 max
53 379206 max
54 379206 max
55 379206 max
56 379206 max
57 379206 max
58 379206 max
59 379206 max
60 379206 max
61 379206 max
62 379206 max
63 379206 max
64 379206 max
65 379206 max
66 379206 max
67 379206 max
68 379206 max
69 379206 max
70 379206 max
71 379206 max
72 379206 max
73 379206 max
74 381283 max
75 381283 max
76 381283 max
77 381283 max
78 381283 max
79 381283 max
80 381283 max
81 381283 max
82 381283 max
83 381283 max
84 381283 max
85 381283 max
86 381283 max
87 381283 max
88 381283 max
89 381283 max
90 381283 max
91 381283 max
92 381283 max
93 381283 max
94 381283 max
95 381283 max
96 381283 max
97 381283 max
98 381283 max
99 381283 max
100 381283 max
101 381283 max
102 381283 max
103 381283 max
104 381283 max
105 381283 max
106 381283 max
107 381283 max
108 381283 max
109 381283 max
110 381283 max
111 381283 max
112 381283 max
113 381283 max
114 381283 max
115 381283 max
116 381283 max
117 381283 max
118 381283 max
119 381283 max
120 381283 max
121 381283 max
122 381283 max
123 381283 max
124 381283 max
125 381283 max
126 381283 max
127 381283 max
128 381283 max
129 381283 max
130 381283 max
131 381283 max
132 381283 max
133 381283 max
134 381283 max
135 381283 max
136 381283 max
137 381283 max
138 381283 max
139 381283 max
140 381283 max
141 381283 max
142 381283 max
143 381283 max
144 381283 max
145 381283 max
146 381283 max
147 381283 max
148 381283 max
149 381283 max
150 381283 max
151 381283 max
152 381283 max
153 381283 max
154 381283 max
155 381283 max
156 381283 max
157 381283 max
158 381283 max
159 381283 max
160 381283 max
161 381283 max
162 381283 max
163 381283 max
164 381283 max
165 381283 max
166 381283 max
167 381283 max
168 381283 max
169 381283 max
170 381283 max
171 381283 max
172 381283 max
173 381283 max
174 381283 max
175 381283 max
176 381283 max
177 381283 max
178 381283 max
179 381283 max
180 381283 max
181 381283 max
182 381283 max
183 381283 max
184 381283 max
185 381283 max
186 381283 max
187 381283 max
188 381283 max
189 381283 max
190 381283 max
191 381283 max
192 381283 max
193 381283 max
194 381283 max
195 381283 max
196 381283 max
197 381283 max
198 381283 max
199 381283 max
200 381283 max
201 381283 max
202 381283 max
203 381283 max
204 381283 max
205 381283 max
206 381283 max
207 381283 max
208 381283 max
209 381283 max
210 381283 max
211 381283 max
212 381283 max
213 381283 max
214 381283 max
215 381283 max
216 381283 max
217 381283 max
218 381283 max
219 381283 max
220 381283 max
221 381283 max
222 381283 max
223 381283 max
224 381283 max
225 381283 max
226 381283 max
227 381283 max
228 381283 max
229 381283 max
230 381283 max
231 381283 max
232 381283 max
233 381283 max
234 381283 max
235 381283 max
236 381283 max
237 381283 max
238 381283 max
239 381283 max
240 381283 max
241 381283 max
242 381283 max
243 381283 max
244 381283 max
245 381283 max
246 381283 max
247 381283 max
248 381283 max
249 381283 max
250 381283 max
251 381283 max
252 381283 max
253 381283 max
254 381283 max
255 381283 max
256 381283 max
257 381283 max
258 381283 max
259 381283 max
260 381283 max
261 381283 max
262 381283 max
263 381283 max
264 381283 max
265 381283 max
266 381283 max
267 381283 max
268 381283 max
269 381283 max
270 381283 max
271 381283 max
272 381283 max
273 381283 max
274 381283 max
275 381283 max
276 381283 max
277 381283 max
278 381283 max
279 381283 max
280 381283 max
281 381283 max
282 381283 max
283 381283 max
284 381283 max
285 381283 max
286 381283 max
287 381283 max
288 381283 max
289 381283 max
290 381283 max
291 381283 max
292 381283 max
293 381283 max
294 381283 max
295 381283 max
296 381283 max
297 381283 max
298 381283 max
299 381283 max
};
\addplot+[scatter,only marks,scatter src=explicit symbolic]table[meta=label] {
x y label
0 177071 med
1 203689 med
2 0 med
3 222387 med
4 246048 med
5 260727 med
6 291266 med
7 304652 med
8 316684 med
9 298856 med
10 310895 med
11 325529 med
12 325482 med
13 321731 med
14 331839 med
15 328313 med
16 334896 med
17 329355 med
18 323019 med
19 333846 med
20 0 med
21 316644 med
22 0 med
23 339992 med
24 342835 med
25 342805 med
26 342835 med
27 340190 med
28 340704 med
29 324668 med
30 337383 med
31 349165 med
32 344293 med
33 324633 med
34 304581 med
35 340190 med
36 325480 med
37 0 med
38 0 med
39 348192 med
40 344691 med
41 349100 med
42 343403 med
43 345092 med
44 334234 med
45 340273 med
46 319303 med
47 0 med
48 0 med
49 0 med
50 325604 med
51 312062 med
52 346244 med
53 319819 med
54 0 med
55 331326 med
56 331272 med
57 312549 med
58 340870 med
59 333786 med
60 339221 med
61 337450 med
62 306591 med
63 0 med
64 326695 med
65 336640 med
66 326920 med
67 324321 med
68 318680 med
69 322584 med
70 340262 med
71 330146 med
72 335143 med
73 325497 med
74 338768 med
75 333000 med
76 334234 med
77 335436 med
78 334852 med
79 328772 med
80 343356 med
81 355122 med
82 347683 med
83 349834 med
84 350694 med
85 337716 med
86 337774 med
87 344868 med
88 323601 med
89 335175 med
90 337038 med
91 0 med
92 318810 med
93 0 med
94 344138 med
95 335875 med
96 350265 med
97 334234 med
98 332342 med
99 332587 med
100 342070 med
101 321539 med
102 335462 med
103 340572 med
104 348307 med
105 336521 med
106 332861 med
107 314439 med
108 0 med
109 322328 med
110 337864 med
111 336001 med
112 348476 med
113 335931 med
114 330240 med
115 345633 med
116 342224 med
117 345069 med
118 340572 med
119 0 med
120 336436 med
121 341654 med
122 321849 med
123 337716 med
124 345765 med
125 349532 med
126 349834 med
127 309110 med
128 336638 med
129 293592 med
130 328578 med
131 345598 med
132 344135 med
133 346690 med
134 342983 med
135 345939 med
136 360185 med
137 308738 med
138 0 med
139 328471 med
140 337439 med
141 328112 med
142 350516 med
143 337203 med
144 300592 med
145 353476 med
146 288533 med
147 318993 med
148 335942 med
149 348380 med
150 326902 med
151 346917 med
152 345074 med
153 322385 med
154 335366 med
155 0 med
156 319546 med
157 331532 med
158 339869 med
159 357042 med
160 355613 med
161 348380 med
162 330740 med
163 0 med
164 306252 med
165 321888 med
166 332861 med
167 0 med
168 332587 med
169 332732 med
170 343193 med
171 321603 med
172 324959 med
173 336393 med
174 331503 med
175 317276 med
176 0 med
177 0 med
178 0 med
179 334561 med
180 340572 med
181 341789 med
182 352037 med
183 338783 med
184 314294 med
185 337152 med
186 337716 med
187 343656 med
188 324421 med
189 333817 med
190 341789 med
191 324959 med
192 331433 med
193 351258 med
194 332448 med
195 332448 med
196 345759 med
197 350704 med
198 355603 med
199 347163 med
200 346643 med
201 341535 med
202 305861 med
203 333351 med
204 358979 med
205 341789 med
206 320440 med
207 345759 med
208 294483 med
209 349178 med
210 340906 med
211 343403 med
212 343403 med
213 335476 med
214 340084 med
215 327524 med
216 330240 med
217 327598 med
218 324959 med
219 0 med
220 318110 med
221 0 med
222 0 med
223 0 med
224 309524 med
225 328231 med
226 344326 med
227 307319 med
228 322800 med
229 342224 med
230 344326 med
231 329465 med
232 327042 med
233 341061 med
234 336701 med
235 336638 med
236 350694 med
237 350056 med
238 333101 med
239 346127 med
240 350056 med
241 356891 med
242 0 med
243 328157 med
244 325194 med
245 331378 med
246 351911 med
247 337800 med
248 318590 med
249 304471 med
250 337976 med
251 321640 med
252 309786 med
253 335116 med
254 331532 med
255 322218 med
256 344483 med
257 345246 med
258 349018 med
259 289567 med
260 331494 med
261 325699 med
262 334798 med
263 340066 med
264 345074 med
265 323150 med
266 327396 med
267 319694 med
268 334694 med
269 340300 med
270 331340 med
271 341109 med
272 353767 med
273 330079 med
274 346484 med
275 350056 med
276 332006 med
277 285002 med
278 321186 med
279 334678 med
280 349067 med
281 336880 med
282 345074 med
283 337152 med
284 334537 med
285 331532 med
286 347838 med
287 341789 med
288 340859 med
289 345759 med
290 338246 med
291 322795 med
292 315702 med
293 330977 med
294 343018 med
295 345882 med
296 352222 med
297 340906 med
298 345765 med
299 327560 med
};
\addplot+[scatter,only marks,scatter src=explicit symbolic]table[meta=label] {
x y label
0 0 min
1 0 min
2 0 min
3 0 min
4 0 min
5 0 min
6 0 min
7 0 min
8 0 min
9 0 min
10 0 min
11 0 min
12 0 min
13 0 min
14 0 min
15 0 min
16 0 min
17 0 min
18 0 min
19 0 min
20 0 min
21 0 min
22 0 min
23 0 min
24 0 min
25 0 min
26 0 min
27 0 min
28 0 min
29 0 min
30 0 min
31 0 min
32 0 min
33 0 min
34 0 min
35 0 min
36 0 min
37 0 min
38 0 min
39 0 min
40 0 min
41 0 min
42 0 min
43 0 min
44 0 min
45 0 min
46 0 min
47 0 min
48 0 min
49 0 min
50 0 min
51 0 min
52 0 min
53 0 min
54 0 min
55 0 min
56 0 min
57 0 min
58 0 min
59 0 min
60 0 min
61 0 min
62 0 min
63 0 min
64 0 min
65 0 min
66 0 min
67 0 min
68 0 min
69 0 min
70 0 min
71 0 min
72 0 min
73 0 min
74 0 min
75 0 min
76 0 min
77 0 min
78 0 min
79 0 min
80 0 min
81 0 min
82 0 min
83 0 min
84 0 min
85 0 min
86 0 min
87 0 min
88 0 min
89 0 min
90 0 min
91 0 min
92 0 min
93 0 min
94 0 min
95 0 min
96 0 min
97 0 min
98 0 min
99 0 min
100 0 min
101 0 min
102 0 min
103 0 min
104 0 min
105 0 min
106 0 min
107 0 min
108 0 min
109 0 min
110 0 min
111 0 min
112 0 min
113 0 min
114 0 min
115 0 min
116 0 min
117 0 min
118 0 min
119 0 min
120 0 min
121 0 min
122 0 min
123 0 min
124 0 min
125 0 min
126 0 min
127 0 min
128 0 min
129 0 min
130 0 min
131 0 min
132 0 min
133 0 min
134 0 min
135 0 min
136 0 min
137 0 min
138 0 min
139 0 min
140 0 min
141 0 min
142 0 min
143 0 min
144 0 min
145 0 min
146 0 min
147 0 min
148 0 min
149 0 min
150 0 min
151 0 min
152 0 min
153 0 min
154 0 min
155 0 min
156 0 min
157 0 min
158 0 min
159 0 min
160 0 min
161 0 min
162 0 min
163 0 min
164 0 min
165 0 min
166 0 min
167 0 min
168 0 min
169 0 min
170 0 min
171 0 min
172 0 min
173 0 min
174 0 min
175 0 min
176 0 min
177 0 min
178 0 min
179 0 min
180 0 min
181 0 min
182 0 min
183 0 min
184 0 min
185 0 min
186 0 min
187 0 min
188 0 min
189 0 min
190 0 min
191 0 min
192 0 min
193 0 min
194 0 min
195 0 min
196 0 min
197 0 min
198 0 min
199 0 min
200 0 min
201 0 min
202 0 min
203 0 min
204 0 min
205 0 min
206 0 min
207 0 min
208 0 min
209 0 min
210 0 min
211 0 min
212 0 min
213 0 min
214 0 min
215 0 min
216 0 min
217 0 min
218 0 min
219 0 min
220 0 min
221 0 min
222 0 min
223 0 min
224 0 min
225 0 min
226 0 min
227 0 min
228 0 min
229 0 min
230 0 min
231 0 min
232 0 min
233 0 min
234 0 min
235 0 min
236 0 min
237 0 min
238 0 min
239 0 min
240 0 min
241 0 min
242 0 min
243 0 min
244 0 min
245 0 min
246 0 min
247 0 min
248 0 min
249 0 min
250 0 min
251 0 min
252 0 min
253 0 min
254 0 min
255 0 min
256 0 min
257 0 min
258 0 min
259 0 min
260 0 min
261 0 min
262 0 min
263 0 min
264 0 min
265 0 min
266 0 min
267 0 min
268 0 min
269 0 min
270 0 min
271 0 min
272 0 min
273 0 min
274 0 min
275 0 min
276 0 min
277 0 min
278 0 min
279 0 min
280 0 min
281 0 min
282 0 min
283 0 min
284 0 min
285 0 min
286 0 min
287 0 min
288 0 min
289 0 min
290 0 min
291 0 min
292 0 min
293 0 min
294 0 min
295 0 min
296 0 min
297 0 min
298 0 min
299 0 min
};


\addplot+[scatter,only marks,scatter src=explicit symbolic]table[meta=label] {
x y label
0 271148 max
1 286759 max
2 286759 max
3 286759 max
4 286759 max
5 286759 max
6 286759 max
7 288860 max
8 288860 max
9 288860 max
10 288860 max
11 288860 max
12 288860 max
13 288860 max
14 300937 max
15 315682 max
16 315682 max
17 315682 max
18 315682 max
19 315682 max
20 315682 max
21 315682 max
22 315682 max
23 315682 max
24 315682 max
25 315682 max
26 315682 max
27 315682 max
28 315682 max
29 327376 max
30 327376 max
31 327376 max
32 327376 max
33 327376 max
34 327376 max
35 327376 max
36 327376 max
37 327376 max
38 327376 max
39 327376 max
40 327376 max
41 327376 max
42 327376 max
43 327376 max
44 327376 max
45 327376 max
46 327376 max
47 327376 max
48 327376 max
49 328950 max
50 328950 max
51 328950 max
52 328950 max
53 328950 max
54 328950 max
55 328950 max
56 328950 max
57 328950 max
58 328950 max
59 328950 max
60 328950 max
61 328950 max
62 328950 max
63 328950 max
64 328950 max
65 346081 max
66 346081 max
67 346081 max
68 346081 max
69 346081 max
70 346081 max
71 346081 max
72 346081 max
73 346081 max
74 346081 max
75 346081 max
76 346081 max
77 346081 max
78 346081 max
79 346081 max
80 346081 max
81 346081 max
82 346081 max
83 346081 max
84 346081 max
85 346081 max
86 346081 max
87 346081 max
88 346081 max
89 346081 max
90 346081 max
91 346081 max
92 346081 max
93 346081 max
94 346081 max
95 346081 max
96 346081 max
97 346081 max
98 346081 max
99 346081 max
100 346081 max
101 346081 max
102 346081 max
103 346081 max
104 365006 max
105 365006 max
106 365006 max
107 365006 max
108 365006 max
109 365006 max
110 365006 max
111 365006 max
112 365006 max
113 365006 max
114 365006 max
115 365006 max
116 365006 max
117 365006 max
118 365006 max
119 365006 max
120 365006 max
121 365006 max
122 365006 max
123 365006 max
124 365006 max
125 365006 max
126 365006 max
127 365006 max
128 365006 max
129 365006 max
130 365006 max
131 365006 max
132 365006 max
133 365006 max
134 365006 max
135 365006 max
136 365006 max
137 365006 max
138 365006 max
139 365006 max
140 365006 max
141 365006 max
142 365006 max
143 365006 max
144 365006 max
145 365006 max
146 365006 max
147 365006 max
148 365006 max
149 365006 max
150 365006 max
151 365006 max
152 365006 max
153 365006 max
154 365006 max
155 365006 max
156 365006 max
157 365006 max
158 365006 max
159 365006 max
160 365006 max
161 365006 max
162 365006 max
163 365006 max
164 365006 max
165 365006 max
166 365006 max
167 365006 max
168 365006 max
169 365006 max
170 365006 max
171 365006 max
172 365006 max
173 365006 max
174 365006 max
175 365006 max
176 365006 max
177 365006 max
178 365006 max
179 365006 max
180 365006 max
181 365006 max
182 365006 max
183 365006 max
184 365006 max
185 365006 max
186 365006 max
187 365006 max
188 365006 max
189 365006 max
190 365006 max
191 365006 max
192 365006 max
193 365006 max
194 365006 max
195 365006 max
196 365006 max
197 365006 max
198 365006 max
199 365006 max
200 365006 max
201 365006 max
202 365006 max
203 365006 max
204 365006 max
205 365006 max
206 365006 max
207 365006 max
208 365006 max
209 365006 max
210 365006 max
211 365006 max
212 365006 max
213 365006 max
214 365006 max
215 365006 max
216 365006 max
217 365006 max
218 365006 max
219 365006 max
220 365006 max
221 365006 max
222 365006 max
223 365006 max
224 365006 max
225 365006 max
226 365006 max
227 365006 max
228 365006 max
229 365006 max
230 365006 max
231 365006 max
232 365006 max
233 365006 max
234 365006 max
235 365006 max
236 365006 max
237 365006 max
238 365006 max
239 365006 max
240 365006 max
241 365006 max
242 365006 max
243 365006 max
244 365006 max
245 365006 max
246 365006 max
247 365006 max
248 365006 max
249 365006 max
250 365006 max
251 365006 max
252 365006 max
253 365006 max
254 365006 max
255 365006 max
256 365006 max
257 365006 max
258 365006 max
259 365006 max
260 365006 max
261 365006 max
262 365006 max
263 365006 max
264 365006 max
265 365006 max
266 365006 max
267 365006 max
268 365006 max
269 365006 max
270 365006 max
271 365006 max
272 365006 max
273 365006 max
274 365006 max
275 365006 max
276 365006 max
277 365006 max
278 365006 max
279 365006 max
280 365006 max
281 365006 max
282 365006 max
283 365006 max
284 365006 max
285 365006 max
286 365006 max
287 365006 max
288 365006 max
289 365006 max
290 365006 max
291 365006 max
292 365006 max
293 365006 max
294 365006 max
295 365006 max
296 365006 max
297 365006 max
298 365006 max
299 365006 max
};
\addplot+[scatter,only marks,scatter src=explicit symbolic]table[meta=label] {
x y label
0 0 med
1 0 med
2 0 med
3 146353 med
4 160156 med
5 0 med
6 0 med
7 213733 med
8 191765 med
9 0 med
10 0 med
11 0 med
12 136824 med
13 191417 med
14 0 med
15 169101 med
16 0 med
17 137413 med
18 188994 med
19 156415 med
20 0 med
21 93485 med
22 0 med
23 0 med
24 177105 med
25 0 med
26 109937 med
27 0 med
28 168142 med
29 0 med
30 179679 med
31 212029 med
32 127074 med
33 0 med
34 176964 med
35 196503 med
36 0 med
37 0 med
38 0 med
39 165947 med
40 0 med
41 0 med
42 0 med
43 212207 med
44 204003 med
45 132162 med
46 163056 med
47 100437 med
48 155517 med
49 192235 med
50 200832 med
51 0 med
52 0 med
53 0 med
54 193078 med
55 149852 med
56 131760 med
57 0 med
58 192900 med
59 0 med
60 0 med
61 170727 med
62 187130 med
63 0 med
64 0 med
65 196906 med
66 166418 med
67 124212 med
68 117684 med
69 188388 med
70 0 med
71 199795 med
72 180451 med
73 175575 med
74 166960 med
75 182037 med
76 187319 med
77 0 med
78 165828 med
79 151136 med
80 0 med
81 153880 med
82 185525 med
83 120497 med
84 0 med
85 217369 med
86 193331 med
87 0 med
88 0 med
89 201365 med
90 0 med
91 143759 med
92 143664 med
93 0 med
94 0 med
95 128574 med
96 0 med
97 158632 med
98 167712 med
99 0 med
100 187275 med
101 0 med
102 0 med
103 0 med
104 0 med
105 155408 med
106 142162 med
107 155218 med
108 165934 med
109 0 med
110 173412 med
111 178208 med
112 99815 med
113 167053 med
114 0 med
115 128516 med
116 152367 med
117 184870 med
118 0 med
119 146030 med
120 154149 med
121 156413 med
122 201878 med
123 177001 med
124 195298 med
125 174056 med
126 136354 med
127 159594 med
128 106801 med
129 160559 med
130 197873 med
131 203751 med
132 57436 med
133 192565 med
134 154210 med
135 0 med
136 179566 med
137 0 med
138 0 med
139 150220 med
140 147688 med
141 117710 med
142 0 med
143 65772 med
144 131806 med
145 0 med
146 0 med
147 126997 med
148 0 med
149 130884 med
150 0 med
151 0 med
152 0 med
153 94174 med
154 146181 med
155 124107 med
156 0 med
157 144025 med
158 0 med
159 0 med
160 106440 med
161 143021 med
162 176585 med
163 157576 med
164 197155 med
165 153883 med
166 168942 med
167 186304 med
168 151488 med
169 193023 med
170 143164 med
171 165507 med
172 200884 med
173 146873 med
174 133349 med
175 187139 med
176 198985 med
177 134410 med
178 169839 med
179 0 med
180 0 med
181 123621 med
182 190770 med
183 169600 med
184 215098 med
185 167068 med
186 128165 med
187 171703 med
188 96284 med
189 148752 med
190 167186 med
191 145272 med
192 216529 med
193 179443 med
194 0 med
195 127301 med
196 0 med
197 0 med
198 126975 med
199 174887 med
200 0 med
201 0 med
202 181333 med
203 164442 med
204 132011 med
205 179078 med
206 171839 med
207 0 med
208 0 med
209 195041 med
210 177987 med
211 100594 med
212 188025 med
213 0 med
214 0 med
215 0 med
216 0 med
217 126040 med
218 166575 med
219 151851 med
220 175899 med
221 150455 med
222 189804 med
223 0 med
224 132362 med
225 172696 med
226 165238 med
227 186061 med
228 55362 med
229 0 med
230 152893 med
231 159825 med
232 165798 med
233 178765 med
234 0 med
235 143979 med
236 0 med
237 173552 med
238 118353 med
239 185108 med
240 166868 med
241 0 med
242 0 med
243 113876 med
244 190141 med
245 0 med
246 0 med
247 0 med
248 0 med
249 0 med
250 148761 med
251 135486 med
252 137938 med
253 148932 med
254 0 med
255 100547 med
256 185911 med
257 0 med
258 177092 med
259 0 med
260 172710 med
261 0 med
262 186717 med
263 189523 med
264 150231 med
265 169709 med
266 148931 med
267 158730 med
268 0 med
269 200457 med
270 152673 med
271 0 med
272 140673 med
273 125089 med
274 0 med
275 128031 med
276 0 med
277 168125 med
278 0 med
279 162397 med
280 208842 med
281 149123 med
282 153905 med
283 169270 med
284 188680 med
285 170090 med
286 0 med
287 168640 med
288 180051 med
289 145605 med
290 180141 med
291 155589 med
292 138787 med
293 141339 med
294 175673 med
295 0 med
296 0 med
297 100995 med
298 179006 med
299 0 med
};
\addplot+[scatter,only marks,scatter src=explicit symbolic]table[meta=label] {
x y label
0 0 min
1 0 min
2 0 min
3 0 min
4 0 min
5 0 min
6 0 min
7 0 min
8 0 min
9 0 min
10 0 min
11 0 min
12 0 min
13 0 min
14 0 min
15 0 min
16 0 min
17 0 min
18 0 min
19 0 min
20 0 min
21 0 min
22 0 min
23 0 min
24 0 min
25 0 min
26 0 min
27 0 min
28 0 min
29 0 min
30 0 min
31 0 min
32 0 min
33 0 min
34 0 min
35 0 min
36 0 min
37 0 min
38 0 min
39 0 min
40 0 min
41 0 min
42 0 min
43 0 min
44 0 min
45 0 min
46 0 min
47 0 min
48 0 min
49 0 min
50 0 min
51 0 min
52 0 min
53 0 min
54 0 min
55 0 min
56 0 min
57 0 min
58 0 min
59 0 min
60 0 min
61 0 min
62 0 min
63 0 min
64 0 min
65 0 min
66 0 min
67 0 min
68 0 min
69 0 min
70 0 min
71 0 min
72 0 min
73 0 min
74 0 min
75 0 min
76 0 min
77 0 min
78 0 min
79 0 min
80 0 min
81 0 min
82 0 min
83 0 min
84 0 min
85 0 min
86 0 min
87 0 min
88 0 min
89 0 min
90 0 min
91 0 min
92 0 min
93 0 min
94 0 min
95 0 min
96 0 min
97 0 min
98 0 min
99 0 min
100 0 min
101 0 min
102 0 min
103 0 min
104 0 min
105 0 min
106 0 min
107 0 min
108 0 min
109 0 min
110 0 min
111 0 min
112 0 min
113 0 min
114 0 min
115 0 min
116 0 min
117 0 min
118 0 min
119 0 min
120 0 min
121 0 min
122 0 min
123 0 min
124 0 min
125 0 min
126 0 min
127 0 min
128 0 min
129 0 min
130 0 min
131 0 min
132 0 min
133 0 min
134 0 min
135 0 min
136 0 min
137 0 min
138 0 min
139 0 min
140 0 min
141 0 min
142 0 min
143 0 min
144 0 min
145 0 min
146 0 min
147 0 min
148 0 min
149 0 min
150 0 min
151 0 min
152 0 min
153 0 min
154 0 min
155 0 min
156 0 min
157 0 min
158 0 min
159 0 min
160 0 min
161 0 min
162 0 min
163 0 min
164 0 min
165 0 min
166 0 min
167 0 min
168 0 min
169 0 min
170 0 min
171 0 min
172 0 min
173 0 min
174 0 min
175 0 min
176 0 min
177 0 min
178 0 min
179 0 min
180 0 min
181 0 min
182 0 min
183 0 min
184 0 min
185 0 min
186 0 min
187 0 min
188 0 min
189 0 min
190 0 min
191 0 min
192 0 min
193 0 min
194 0 min
195 0 min
196 0 min
197 0 min
198 0 min
199 0 min
200 0 min
201 0 min
202 0 min
203 0 min
204 0 min
205 0 min
206 0 min
207 0 min
208 0 min
209 0 min
210 0 min
211 0 min
212 0 min
213 0 min
214 0 min
215 0 min
216 0 min
217 0 min
218 0 min
219 0 min
220 0 min
221 0 min
222 0 min
223 0 min
224 0 min
225 0 min
226 0 min
227 0 min
228 0 min
229 0 min
230 0 min
231 0 min
232 0 min
233 0 min
234 0 min
235 0 min
236 0 min
237 0 min
238 0 min
239 0 min
240 0 min
241 0 min
242 0 min
243 0 min
244 0 min
245 0 min
246 0 min
247 0 min
248 0 min
249 0 min
250 0 min
251 0 min
252 0 min
253 0 min
254 0 min
255 0 min
256 0 min
257 0 min
258 0 min
259 0 min
260 0 min
261 0 min
262 0 min
263 0 min
264 0 min
265 0 min
266 0 min
267 0 min
268 0 min
269 0 min
270 0 min
271 0 min
272 0 min
273 0 min
274 0 min
275 0 min
276 0 min
277 0 min
278 0 min
279 0 min
280 0 min
281 0 min
282 0 min
283 0 min
284 0 min
285 0 min
286 0 min
287 0 min
288 0 min
289 0 min
290 0 min
291 0 min
292 0 min
293 0 min
294 0 min
295 0 min
296 0 min
297 0 min
298 0 min
299 0 min
};


\addplot+[scatter,only marks,scatter src=explicit symbolic]table[meta=label] {
x y label
0 315921 max
1 329608 max
2 338662 max
3 338662 max
4 338662 max
5 338662 max
6 338662 max
7 338662 max
8 338662 max
9 338662 max
10 354388 max
11 354388 max
12 354388 max
13 354388 max
14 354388 max
15 354388 max
16 354388 max
17 354388 max
18 354388 max
19 354388 max
20 354388 max
21 354388 max
22 354388 max
23 354388 max
24 354388 max
25 354388 max
26 354388 max
27 354388 max
28 354388 max
29 354388 max
30 354388 max
31 354388 max
32 354388 max
33 354388 max
34 354388 max
35 354388 max
36 354388 max
37 354388 max
38 354388 max
39 354388 max
40 354388 max
41 354388 max
42 354388 max
43 354388 max
44 354388 max
45 354388 max
46 354388 max
47 354388 max
48 354388 max
49 354388 max
50 354388 max
51 354388 max
52 354388 max
53 354388 max
54 354388 max
55 354388 max
56 354388 max
57 354388 max
58 354388 max
59 354388 max
60 354388 max
61 354388 max
62 354388 max
63 354388 max
64 354388 max
65 354388 max
66 354388 max
67 354388 max
68 354388 max
69 354388 max
70 354388 max
71 354388 max
72 354388 max
73 354388 max
74 354388 max
75 354388 max
76 354388 max
77 354388 max
78 354388 max
79 354388 max
80 354388 max
81 354388 max
82 354388 max
83 354388 max
84 354388 max
85 354388 max
86 354388 max
87 354388 max
88 354388 max
89 354388 max
90 354388 max
91 354388 max
92 354388 max
93 354388 max
94 354388 max
95 354388 max
96 354388 max
97 354388 max
98 354388 max
99 354388 max
100 354388 max
101 354388 max
102 354388 max
103 354388 max
104 354388 max
105 354388 max
106 354388 max
107 354388 max
108 354388 max
109 354388 max
110 354388 max
111 354388 max
112 354388 max
113 354388 max
114 354388 max
115 354388 max
116 354388 max
117 354388 max
118 354388 max
119 354388 max
120 354388 max
121 354388 max
122 354388 max
123 354388 max
124 354388 max
125 354388 max
126 354388 max
127 354388 max
128 354388 max
129 354388 max
130 354388 max
131 354388 max
132 354388 max
133 354388 max
134 354388 max
135 354388 max
136 354388 max
137 354388 max
138 354388 max
139 354388 max
140 354388 max
141 354388 max
142 354388 max
143 354388 max
144 354388 max
145 354388 max
146 354388 max
147 354388 max
148 354388 max
149 354388 max
150 354388 max
151 354388 max
152 354388 max
153 354388 max
154 354388 max
155 354388 max
156 354388 max
157 354388 max
158 354388 max
159 354388 max
160 354388 max
161 354388 max
162 354388 max
163 354388 max
164 354388 max
165 354388 max
166 354388 max
167 354388 max
168 354388 max
169 354388 max
170 354388 max
171 354388 max
172 354388 max
173 354388 max
174 354388 max
175 354388 max
176 354388 max
177 354388 max
178 354388 max
179 355269 max
180 355269 max
181 355269 max
182 355269 max
183 355269 max
184 355269 max
185 355269 max
186 355269 max
187 355269 max
188 355269 max
189 355269 max
190 355269 max
191 355269 max
192 355269 max
193 355269 max
194 355269 max
195 355269 max
196 355269 max
197 355269 max
198 355269 max
199 355269 max
200 355269 max
201 355269 max
202 355269 max
203 355269 max
204 355269 max
205 355269 max
206 355269 max
207 355269 max
208 355269 max
209 355269 max
210 355269 max
211 355269 max
212 355269 max
213 355269 max
214 355269 max
215 355269 max
216 355269 max
217 355269 max
218 355269 max
219 355269 max
220 355269 max
221 355269 max
222 355269 max
223 355269 max
224 355269 max
225 355269 max
226 355269 max
227 355269 max
228 355269 max
229 355269 max
230 355269 max
231 355269 max
232 355269 max
233 355269 max
234 355269 max
235 355269 max
236 355269 max
237 355269 max
238 355269 max
239 355269 max
240 355269 max
241 355269 max
242 355269 max
243 355269 max
244 355269 max
245 355269 max
246 355269 max
247 355269 max
248 355269 max
249 355269 max
250 355269 max
251 355269 max
252 355269 max
253 355269 max
254 355269 max
255 355269 max
256 355269 max
257 355269 max
258 355269 max
259 355269 max
260 355269 max
261 355269 max
262 355269 max
263 355269 max
264 355269 max
265 355269 max
266 355269 max
267 355269 max
268 355269 max
269 355269 max
270 355269 max
271 355269 max
272 355269 max
273 355269 max
274 355269 max
275 355269 max
276 355269 max
277 355269 max
278 355269 max
279 355269 max
280 355269 max
281 355269 max
282 355269 max
283 355269 max
284 355269 max
285 355269 max
286 355269 max
287 355269 max
288 355269 max
289 355269 max
290 355269 max
291 355269 max
292 355269 max
293 355269 max
294 355269 max
295 355269 max
296 355269 max
297 355269 max
298 355269 max
299 355269 max
};
\addplot+[scatter,only marks,scatter src=explicit symbolic]table[meta=label] {
x y label
0 152738 med
1 0 med
2 0 med
3 0 med
4 0 med
5 0 med
6 0 med
7 0 med
8 0 med
9 0 med
10 0 med
11 0 med
12 121629 med
13 0 med
14 0 med
15 0 med
16 0 med
17 0 med
18 0 med
19 124054 med
20 0 med
21 0 med
22 0 med
23 0 med
24 0 med
25 0 med
26 0 med
27 0 med
28 0 med
29 0 med
30 0 med
31 0 med
32 0 med
33 0 med
34 0 med
35 0 med
36 0 med
37 113357 med
38 0 med
39 0 med
40 0 med
41 0 med
42 0 med
43 0 med
44 0 med
45 0 med
46 0 med
47 113033 med
48 145583 med
49 0 med
50 0 med
51 0 med
52 0 med
53 0 med
54 0 med
55 0 med
56 0 med
57 0 med
58 0 med
59 0 med
60 0 med
61 0 med
62 0 med
63 0 med
64 0 med
65 123526 med
66 0 med
67 137791 med
68 0 med
69 0 med
70 0 med
71 0 med
72 0 med
73 0 med
74 0 med
75 0 med
76 0 med
77 141142 med
78 0 med
79 0 med
80 162005 med
81 0 med
82 140483 med
83 0 med
84 0 med
85 0 med
86 0 med
87 0 med
88 0 med
89 0 med
90 0 med
91 0 med
92 0 med
93 0 med
94 0 med
95 0 med
96 0 med
97 0 med
98 0 med
99 0 med
100 0 med
101 0 med
102 0 med
103 0 med
104 0 med
105 110341 med
106 0 med
107 117163 med
108 0 med
109 0 med
110 0 med
111 0 med
112 0 med
113 0 med
114 0 med
115 0 med
116 0 med
117 0 med
118 0 med
119 0 med
120 0 med
121 0 med
122 0 med
123 0 med
124 0 med
125 0 med
126 0 med
127 0 med
128 0 med
129 0 med
130 0 med
131 0 med
132 0 med
133 121748 med
134 0 med
135 0 med
136 0 med
137 0 med
138 0 med
139 0 med
140 0 med
141 0 med
142 0 med
143 0 med
144 88899 med
145 0 med
146 0 med
147 118694 med
148 132809 med
149 0 med
150 0 med
151 0 med
152 0 med
153 139866 med
154 0 med
155 0 med
156 0 med
157 0 med
158 0 med
159 0 med
160 0 med
161 0 med
162 0 med
163 0 med
164 0 med
165 122428 med
166 0 med
167 0 med
168 0 med
169 0 med
170 0 med
171 0 med
172 0 med
173 0 med
174 0 med
175 0 med
176 126689 med
177 0 med
178 0 med
179 0 med
180 0 med
181 119821 med
182 0 med
183 91239 med
184 0 med
185 0 med
186 0 med
187 0 med
188 0 med
189 0 med
190 0 med
191 0 med
192 0 med
193 0 med
194 0 med
195 0 med
196 0 med
197 0 med
198 0 med
199 0 med
200 0 med
201 0 med
202 0 med
203 0 med
204 0 med
205 0 med
206 0 med
207 0 med
208 0 med
209 0 med
210 0 med
211 122786 med
212 0 med
213 0 med
214 0 med
215 0 med
216 0 med
217 145380 med
218 0 med
219 0 med
220 0 med
221 0 med
222 0 med
223 0 med
224 143636 med
225 0 med
226 0 med
227 118708 med
228 0 med
229 0 med
230 0 med
231 108453 med
232 0 med
233 0 med
234 0 med
235 105037 med
236 0 med
237 93779 med
238 0 med
239 0 med
240 0 med
241 0 med
242 142029 med
243 0 med
244 0 med
245 0 med
246 0 med
247 0 med
248 0 med
249 0 med
250 0 med
251 118194 med
252 118623 med
253 0 med
254 0 med
255 0 med
256 0 med
257 0 med
258 0 med
259 0 med
260 0 med
261 0 med
262 0 med
263 121939 med
264 0 med
265 0 med
266 0 med
267 0 med
268 0 med
269 0 med
270 0 med
271 0 med
272 0 med
273 0 med
274 0 med
275 0 med
276 0 med
277 0 med
278 0 med
279 0 med
280 0 med
281 0 med
282 0 med
283 0 med
284 0 med
285 0 med
286 136411 med
287 0 med
288 0 med
289 115255 med
290 0 med
291 0 med
292 0 med
293 0 med
294 0 med
295 0 med
296 0 med
297 0 med
298 0 med
299 0 med
};
\addplot+[scatter,only marks,scatter src=explicit symbolic]table[meta=label] {
x y label
0 0 min
1 0 min
2 0 min
3 0 min
4 0 min
5 0 min
6 0 min
7 0 min
8 0 min
9 0 min
10 0 min
11 0 min
12 0 min
13 0 min
14 0 min
15 0 min
16 0 min
17 0 min
18 0 min
19 0 min
20 0 min
21 0 min
22 0 min
23 0 min
24 0 min
25 0 min
26 0 min
27 0 min
28 0 min
29 0 min
30 0 min
31 0 min
32 0 min
33 0 min
34 0 min
35 0 min
36 0 min
37 0 min
38 0 min
39 0 min
40 0 min
41 0 min
42 0 min
43 0 min
44 0 min
45 0 min
46 0 min
47 0 min
48 0 min
49 0 min
50 0 min
51 0 min
52 0 min
53 0 min
54 0 min
55 0 min
56 0 min
57 0 min
58 0 min
59 0 min
60 0 min
61 0 min
62 0 min
63 0 min
64 0 min
65 0 min
66 0 min
67 0 min
68 0 min
69 0 min
70 0 min
71 0 min
72 0 min
73 0 min
74 0 min
75 0 min
76 0 min
77 0 min
78 0 min
79 0 min
80 0 min
81 0 min
82 0 min
83 0 min
84 0 min
85 0 min
86 0 min
87 0 min
88 0 min
89 0 min
90 0 min
91 0 min
92 0 min
93 0 min
94 0 min
95 0 min
96 0 min
97 0 min
98 0 min
99 0 min
100 0 min
101 0 min
102 0 min
103 0 min
104 0 min
105 0 min
106 0 min
107 0 min
108 0 min
109 0 min
110 0 min
111 0 min
112 0 min
113 0 min
114 0 min
115 0 min
116 0 min
117 0 min
118 0 min
119 0 min
120 0 min
121 0 min
122 0 min
123 0 min
124 0 min
125 0 min
126 0 min
127 0 min
128 0 min
129 0 min
130 0 min
131 0 min
132 0 min
133 0 min
134 0 min
135 0 min
136 0 min
137 0 min
138 0 min
139 0 min
140 0 min
141 0 min
142 0 min
143 0 min
144 0 min
145 0 min
146 0 min
147 0 min
148 0 min
149 0 min
150 0 min
151 0 min
152 0 min
153 0 min
154 0 min
155 0 min
156 0 min
157 0 min
158 0 min
159 0 min
160 0 min
161 0 min
162 0 min
163 0 min
164 0 min
165 0 min
166 0 min
167 0 min
168 0 min
169 0 min
170 0 min
171 0 min
172 0 min
173 0 min
174 0 min
175 0 min
176 0 min
177 0 min
178 0 min
179 0 min
180 0 min
181 0 min
182 0 min
183 0 min
184 0 min
185 0 min
186 0 min
187 0 min
188 0 min
189 0 min
190 0 min
191 0 min
192 0 min
193 0 min
194 0 min
195 0 min
196 0 min
197 0 min
198 0 min
199 0 min
200 0 min
201 0 min
202 0 min
203 0 min
204 0 min
205 0 min
206 0 min
207 0 min
208 0 min
209 0 min
210 0 min
211 0 min
212 0 min
213 0 min
214 0 min
215 0 min
216 0 min
217 0 min
218 0 min
219 0 min
220 0 min
221 0 min
222 0 min
223 0 min
224 0 min
225 0 min
226 0 min
227 0 min
228 0 min
229 0 min
230 0 min
231 0 min
232 0 min
233 0 min
234 0 min
235 0 min
236 0 min
237 0 min
238 0 min
239 0 min
240 0 min
241 0 min
242 0 min
243 0 min
244 0 min
245 0 min
246 0 min
247 0 min
248 0 min
249 0 min
250 0 min
251 0 min
252 0 min
253 0 min
254 0 min
255 0 min
256 0 min
257 0 min
258 0 min
259 0 min
260 0 min
261 0 min
262 0 min
263 0 min
264 0 min
265 0 min
266 0 min
267 0 min
268 0 min
269 0 min
270 0 min
271 0 min
272 0 min
273 0 min
274 0 min
275 0 min
276 0 min
277 0 min
278 0 min
279 0 min
280 0 min
281 0 min
282 0 min
283 0 min
284 0 min
285 0 min
286 0 min
287 0 min
288 0 min
289 0 min
290 0 min
291 0 min
292 0 min
293 0 min
294 0 min
295 0 min
296 0 min
297 0 min
298 0 min
299 0 min
};



\section{Variation in configuration variables}
When we take a look at \cref{plot:genProfile30-pool300} we can see the higher
the mutation coefficient is, the more spread out are our configurations, as
the chance for the breeding result to be completely random grows.

If we decide to use a larger pool for our algorithm, as you can see in \cref{plot:genProfile30-pool300},
where we used 300 elements, we gain the result in earlier generations. Moreover,
due to the nature of our fitness function, many of the medians have a fitness value of zero.

\section{Conclusion}
In conclusion, we have created and tested the genetic algorithm for the 0/1 version
of the knapsack problem. We have managed to calculate the result with the maximal
measured imprecision of 1\% with considerably low time complexity, especially for
greater instances of the knapsack problem.

\newpage
\bibliographystyle{iso690}
% \nocite{*} % all entries in the bib file
\bibliography{database.bib}

\end{document}