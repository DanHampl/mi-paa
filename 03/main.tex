
\NeedsTeXFormat{LaTeX2e}
\LoadClass{article}
\RequirePackage{todonotes}
\RequirePackage[parfill]{parskip}
\RequirePackage[margin=2.8cm]{geometry}
\RequirePackage{hyperref}
\RequirePackage[english]{babel}
\RequirePackage{pgfplots}
\RequirePackage{listings}
\RequirePackage{amsfonts}
\RequirePackage[noabbrev,capitalize,nameinlink]{cleveref}
\RequirePackage{subfiles}
\RequirePackage{mathtools}

\providecommand{\tightlist}{%
  \setlength{\itemsep}{0pt}\setlength{\parskip}{0pt}}

\begin{document}
\title{\textbf{MI-PAA~--~Task 3}\\
Comparison of various algoritms solving the optimisation version of the 0/1 knapsack problem}
\author{Daniel Hampl (hampldan)}
\date{\today}
\maketitle

\tableofcontents
\newpage

\section{Introduction}
The knapsack problem is one of the most widespread NP-Complete problems. It can be described as having a knapsack with a limited capacity and multiple items, where each of these items has a set value and weight. Our task is to fill the knapsack with items of highest combined value possible.

In this paper, we will be focussing approximation algorithms solving the knapsack problem and the deviation of results provided by these approximation algorithms.

\subsection{Definition\cite{WEBSITE:knapsackDef}}
We have weight $W$ and $n$ items, where item $i$ is described as a pair $(w_i, v_i)$.

\begin{itemize}
    \item $w_i$ is a weight of object $n_i$
    \item $v_i$ is a value of object $n_i$
    \item $w_i, v_i \in \mathbb{N}\setminus\{0\}$
\end{itemize}

Find value $V$, where:

\begin{itemize}
    \item $\sum_i(x_i*v_i) = V$
    \item $\sum_i(x_i*w_i) \leq W$
    \item $x_i \in \{0,1\} \forall i$
\end{itemize}

\subsection{Task}
Compare various algorithms for solving the optimisation version of the 0/1 knapsack problem
with respect to their efficiency and imprecision of the approximation algorithms. Moreover,
observe the dependency of those algorithms on different datasets.

\begin{itemize}
    \item Dynamic programing with decomposition by value or weight
    \item Simple greedy heuristic approximation
    \item Modification of this greedy heuristics, where only the single most valuable item is selected
    \item FPTAS algorithm
\end{itemize}

Experimentally evaluate the dependence of computational complexity on instance size. And deviation of results provided by approximation algorithms as compared to the exact results.


\addplot[scatter,scatter src=explicit symbolic]table[meta=label] {
x y label
4 .860686 hungryStupidDeviation
10 .566400 hungryStupidDeviation
15 .286770 hungryStupidDeviation
20 .276166 hungryStupidDeviation
};
\addplot[scatter,scatter src=explicit symbolic]table[meta=label] {
x y label
4 1.493548 singleDeviation
10 .566400 singleDeviation
15 .286770 singleDeviation
20 .276166 singleDeviation
};
\addplot[scatter,scatter src=explicit symbolic]table[meta=label] {
x y label
4 5.959800 fptasDeviation
10 3.533680 fptasDeviation
15 2.698540 fptasDeviation
20 1.992056 fptasDeviation
};


\addplot[scatter,scatter src=explicit symbolic]table[meta=label] {
x y label
4 2.158840 hungryStupidDeviation
10 .356700 hungryStupidDeviation
15 .098076 hungryStupidDeviation
20 .059370 hungryStupidDeviation
};
\addplot[scatter,scatter src=explicit symbolic]table[meta=label] {
x y label
4 2.701980 singleDeviation
10 .356700 singleDeviation
15 .098076 singleDeviation
20 .059370 singleDeviation
};
\addplot[scatter,scatter src=explicit symbolic]table[meta=label] {
x y label
4 5.846760 fptasDeviation
10 3.800300 fptasDeviation
15 2.905380 fptasDeviation
20 2.171120 fptasDeviation
};


\addplot[scatter,scatter src=explicit symbolic]table[meta=label] {
x y label
4 .000012 solvePriceDynamic
10 .000466 solvePriceDynamic
15 .016540 solvePriceDynamic
20 .562922 solvePriceDynamic
};
\addplot[scatter,scatter src=explicit symbolic]table[meta=label] {
x y label
4 .000002 solveHungry
10 .000006 solveHungry
15 .000008 solveHungry
20 .000012 solveHungry
};
\addplot[scatter,scatter src=explicit symbolic]table[meta=label] {
x y label
4 .000002 solveSingle
10 .000006 solveSingle
15 .000006 solveSingle
20 .000012 solveSingle
};
\addplot[scatter,scatter src=explicit symbolic]table[meta=label] {
x y label
4 .000012 fptas
10 .000156 fptas
15 .000640 fptas
20 .002398 fptas
};
\addplot[scatter,scatter src=explicit symbolic]table[meta=label] {
x y label
4 .000024 solveStupid
10 .001344 solveStupid
15 .042402 solveStupid
20 1.636722 solveStupid
};
\addplot[scatter,scatter src=explicit symbolic]table[meta=label] {
x y label
4 .000018 solveSmart
10 .000232 solveSmart
15 .002240 solveSmart
20 .030288 solveSmart
};

\begin{figure}
	\centering
	\pgfplotsset{every axis legend/.append style={
		at={(1.05,0.5)},
		anchor=west}}
	\begin{tikzpicture}
		\begin{axis}[
			xlabel=Items in set,
			ylabel=Average imprecision in \%,
			scatter/classes={
				hungryStupidDeviation={mark=square*,blue},
				singleDeviation={mark=square*,green},
				fptasDeviation={mark=square*,red}
				}
            ]
            \begin{figure}
	\centering
	\pgfplotsset{every axis legend/.append style={
		at={(1.05,0.5)},
		anchor=west}}
	\begin{tikzpicture}
		\begin{axis}[
			xlabel=Items in set,
			ylabel=Average imprecision in \%,
			scatter/classes={
				hungryStupidDeviation={mark=square*,blue},
				singleDeviation={mark=square*,green},
				fptasDeviation={mark=square*,red}
				}
            ]
            \begin{figure}
	\centering
	\pgfplotsset{every axis legend/.append style={
		at={(1.05,0.5)},
		anchor=west}}
	\begin{tikzpicture}
		\begin{axis}[
			xlabel=Items in set,
			ylabel=Average imprecision in \%,
			scatter/classes={
				hungryStupidDeviation={mark=square*,blue},
				singleDeviation={mark=square*,green},
				fptasDeviation={mark=square*,red}
				}
            ]
            \input{data/tex/deviation/correlationWeak.tex}
			\addlegendentry{Greedy heuristics}
			\addlegendentry{Modified greedy heuristics}
			\addlegendentry{FPTAS}
		\end{axis}
	\end{tikzpicture}
\caption{Average imprecision on data with weak correlation}
\label{plot:correlationWeakDeviation}
\end{figure}

			\addlegendentry{Greedy heuristics}
			\addlegendentry{Modified greedy heuristics}
			\addlegendentry{FPTAS}
		\end{axis}
	\end{tikzpicture}
\caption{Average imprecision on data with weak correlation}
\label{plot:correlationWeakDeviation}
\end{figure}

			\addlegendentry{Greedy heuristics}
			\addlegendentry{Modified greedy heuristics}
			\addlegendentry{FPTAS}
		\end{axis}
	\end{tikzpicture}
\caption{Average imprecision on data with weak correlation}
\label{plot:correlationWeakDeviation}
\end{figure}


\addplot[scatter,scatter src=explicit symbolic]table[meta=label] {
x y label
4 .000012 solvePriceDynamic
10 .000482 solvePriceDynamic
15 .017292 solvePriceDynamic
20 .542394 solvePriceDynamic
};
\addplot[scatter,scatter src=explicit symbolic]table[meta=label] {
x y label
4 .000002 solveHungry
10 .000004 solveHungry
15 .000008 solveHungry
20 .000010 solveHungry
};
\addplot[scatter,scatter src=explicit symbolic]table[meta=label] {
x y label
4 .000002 solveSingle
10 .000004 solveSingle
15 .000008 solveSingle
20 .000012 solveSingle
};
\addplot[scatter,scatter src=explicit symbolic]table[meta=label] {
x y label
4 .000010 fptas
10 .000144 fptas
15 .000630 fptas
20 .002100 fptas
};
\addplot[scatter,scatter src=explicit symbolic]table[meta=label] {
x y label
4 .000022 solveStupid
10 .001332 solveStupid
15 .043850 solveStupid
20 1.608816 solveStupid
};
\addplot[scatter,scatter src=explicit symbolic]table[meta=label] {
x y label
4 .000014 solveSmart
10 .000076 solveSmart
15 .000264 solveSmart
20 .001012 solveSmart
};


\addplot[scatter,scatter src=explicit symbolic]table[meta=label] {
x y label
4 .000012 solvePriceDynamic
10 .000548 solvePriceDynamic
15 .016784 solvePriceDynamic
20 .494500 solvePriceDynamic
};
\addplot[scatter,scatter src=explicit symbolic]table[meta=label] {
x y label
4 .000002 solveHungry
10 .000006 solveHungry
15 .000008 solveHungry
20 .000010 solveHungry
};
\addplot[scatter,scatter src=explicit symbolic]table[meta=label] {
x y label
4 .000002 solveSingle
10 .000006 solveSingle
15 .000008 solveSingle
20 .000010 solveSingle
};
\addplot[scatter,scatter src=explicit symbolic]table[meta=label] {
x y label
4 .000010 fptas
10 .000170 fptas
15 .000644 fptas
20 .002006 fptas
};
\addplot[scatter,scatter src=explicit symbolic]table[meta=label] {
x y label
4 .000022 solveStupid
10 .001562 solveStupid
15 .043078 solveStupid
20 1.507664 solveStupid
};
\addplot[scatter,scatter src=explicit symbolic]table[meta=label] {
x y label
4 .000016 solveSmart
10 .000092 solveSmart
15 .000268 solveSmart
20 .001024 solveSmart
};


\addplot[scatter,scatter src=explicit symbolic]table[meta=label] {
x y label
4 .000010 solvePriceDynamic
10 .000328 solvePriceDynamic
15 .001646 solvePriceDynamic
20 .004958 solvePriceDynamic
};
\addplot[scatter,scatter src=explicit symbolic]table[meta=label] {
x y label
4 .000002 solveHungry
10 .000004 solveHungry
15 .000006 solveHungry
20 .000010 solveHungry
};
\addplot[scatter,scatter src=explicit symbolic]table[meta=label] {
x y label
4 .000002 solveSingle
10 .000004 solveSingle
15 .000006 solveSingle
20 .000010 solveSingle
};
\addplot[scatter,scatter src=explicit symbolic]table[meta=label] {
x y label
4 .000010 fptas
10 .000140 fptas
15 .000546 fptas
20 .001994 fptas
};
\addplot[scatter,scatter src=explicit symbolic]table[meta=label] {
x y label
4 .000022 solveStupid
10 .001338 solveStupid
15 .041064 solveStupid
20 1.669862 solveStupid
};
\addplot[scatter,scatter src=explicit symbolic]table[meta=label] {
x y label
4 .000014 solveSmart
10 .000080 solveSmart
15 .000244 solveSmart
20 .000956 solveSmart
};


\addplot[scatter,scatter src=explicit symbolic]table[meta=label] {
x y label
4 3.911340 hungryStupidDeviation
10 2.209160 hungryStupidDeviation
15 1.676708 hungryStupidDeviation
20 .858728 hungryStupidDeviation
};
\addplot[scatter,scatter src=explicit symbolic]table[meta=label] {
x y label
4 31.760600 singleDeviation
10 2.209160 singleDeviation
15 1.676708 singleDeviation
20 .858728 singleDeviation
};
\addplot[scatter,scatter src=explicit symbolic]table[meta=label] {
x y label
4 6.724640 fptasDeviation
10 2.920400 fptasDeviation
15 2.143400 fptasDeviation
20 1.608108 fptasDeviation
};


\addplot[scatter,scatter src=explicit symbolic]table[meta=label] {
x y label
4 3.829840 hungryStupidDeviation
10 2.125580 hungryStupidDeviation
15 1.699950 hungryStupidDeviation
20 .905228 hungryStupidDeviation
};
\addplot[scatter,scatter src=explicit symbolic]table[meta=label] {
x y label
4 32.843600 singleDeviation
10 2.125580 singleDeviation
15 1.699950 singleDeviation
20 .905228 singleDeviation
};
\addplot[scatter,scatter src=explicit symbolic]table[meta=label] {
x y label
4 6.585160 fptasDeviation
10 2.922120 fptasDeviation
15 2.279220 fptasDeviation
20 1.654382 fptasDeviation
};



\addplot[scatter,scatter src=explicit symbolic]table[meta=label] {
x y label
4 .860686 hungryStupidDeviation
10 .566400 hungryStupidDeviation
15 .286770 hungryStupidDeviation
20 .276166 hungryStupidDeviation
};
\addplot[scatter,scatter src=explicit symbolic]table[meta=label] {
x y label
4 1.493548 singleDeviation
10 .566400 singleDeviation
15 .286770 singleDeviation
20 .276166 singleDeviation
};
\addplot[scatter,scatter src=explicit symbolic]table[meta=label] {
x y label
4 5.959800 fptasDeviation
10 3.533680 fptasDeviation
15 2.698540 fptasDeviation
20 1.992056 fptasDeviation
};


\addplot[scatter,scatter src=explicit symbolic]table[meta=label] {
x y label
4 2.158840 hungryStupidDeviation
10 .356700 hungryStupidDeviation
15 .098076 hungryStupidDeviation
20 .059370 hungryStupidDeviation
};
\addplot[scatter,scatter src=explicit symbolic]table[meta=label] {
x y label
4 2.701980 singleDeviation
10 .356700 singleDeviation
15 .098076 singleDeviation
20 .059370 singleDeviation
};
\addplot[scatter,scatter src=explicit symbolic]table[meta=label] {
x y label
4 5.846760 fptasDeviation
10 3.800300 fptasDeviation
15 2.905380 fptasDeviation
20 2.171120 fptasDeviation
};


\addplot[scatter,scatter src=explicit symbolic]table[meta=label] {
x y label
4 .000012 solvePriceDynamic
10 .000466 solvePriceDynamic
15 .016540 solvePriceDynamic
20 .562922 solvePriceDynamic
};
\addplot[scatter,scatter src=explicit symbolic]table[meta=label] {
x y label
4 .000002 solveHungry
10 .000006 solveHungry
15 .000008 solveHungry
20 .000012 solveHungry
};
\addplot[scatter,scatter src=explicit symbolic]table[meta=label] {
x y label
4 .000002 solveSingle
10 .000006 solveSingle
15 .000006 solveSingle
20 .000012 solveSingle
};
\addplot[scatter,scatter src=explicit symbolic]table[meta=label] {
x y label
4 .000012 fptas
10 .000156 fptas
15 .000640 fptas
20 .002398 fptas
};
\addplot[scatter,scatter src=explicit symbolic]table[meta=label] {
x y label
4 .000024 solveStupid
10 .001344 solveStupid
15 .042402 solveStupid
20 1.636722 solveStupid
};
\addplot[scatter,scatter src=explicit symbolic]table[meta=label] {
x y label
4 .000018 solveSmart
10 .000232 solveSmart
15 .002240 solveSmart
20 .030288 solveSmart
};

\begin{figure}
	\centering
	\pgfplotsset{every axis legend/.append style={
		at={(1.05,0.5)},
		anchor=west}}
	\begin{tikzpicture}
		\begin{axis}[
			xlabel=Items in set,
			ylabel=Average imprecision in \%,
			scatter/classes={
				hungryStupidDeviation={mark=square*,blue},
				singleDeviation={mark=square*,green},
				fptasDeviation={mark=square*,red}
				}
            ]
            \begin{figure}
	\centering
	\pgfplotsset{every axis legend/.append style={
		at={(1.05,0.5)},
		anchor=west}}
	\begin{tikzpicture}
		\begin{axis}[
			xlabel=Items in set,
			ylabel=Average imprecision in \%,
			scatter/classes={
				hungryStupidDeviation={mark=square*,blue},
				singleDeviation={mark=square*,green},
				fptasDeviation={mark=square*,red}
				}
            ]
            \begin{figure}
	\centering
	\pgfplotsset{every axis legend/.append style={
		at={(1.05,0.5)},
		anchor=west}}
	\begin{tikzpicture}
		\begin{axis}[
			xlabel=Items in set,
			ylabel=Average imprecision in \%,
			scatter/classes={
				hungryStupidDeviation={mark=square*,blue},
				singleDeviation={mark=square*,green},
				fptasDeviation={mark=square*,red}
				}
            ]
            \input{data/tex/deviation/correlationWeak.tex}
			\addlegendentry{Greedy heuristics}
			\addlegendentry{Modified greedy heuristics}
			\addlegendentry{FPTAS}
		\end{axis}
	\end{tikzpicture}
\caption{Average imprecision on data with weak correlation}
\label{plot:correlationWeakDeviation}
\end{figure}

			\addlegendentry{Greedy heuristics}
			\addlegendentry{Modified greedy heuristics}
			\addlegendentry{FPTAS}
		\end{axis}
	\end{tikzpicture}
\caption{Average imprecision on data with weak correlation}
\label{plot:correlationWeakDeviation}
\end{figure}

			\addlegendentry{Greedy heuristics}
			\addlegendentry{Modified greedy heuristics}
			\addlegendentry{FPTAS}
		\end{axis}
	\end{tikzpicture}
\caption{Average imprecision on data with weak correlation}
\label{plot:correlationWeakDeviation}
\end{figure}


\addplot[scatter,scatter src=explicit symbolic]table[meta=label] {
x y label
4 .000012 solvePriceDynamic
10 .000482 solvePriceDynamic
15 .017292 solvePriceDynamic
20 .542394 solvePriceDynamic
};
\addplot[scatter,scatter src=explicit symbolic]table[meta=label] {
x y label
4 .000002 solveHungry
10 .000004 solveHungry
15 .000008 solveHungry
20 .000010 solveHungry
};
\addplot[scatter,scatter src=explicit symbolic]table[meta=label] {
x y label
4 .000002 solveSingle
10 .000004 solveSingle
15 .000008 solveSingle
20 .000012 solveSingle
};
\addplot[scatter,scatter src=explicit symbolic]table[meta=label] {
x y label
4 .000010 fptas
10 .000144 fptas
15 .000630 fptas
20 .002100 fptas
};
\addplot[scatter,scatter src=explicit symbolic]table[meta=label] {
x y label
4 .000022 solveStupid
10 .001332 solveStupid
15 .043850 solveStupid
20 1.608816 solveStupid
};
\addplot[scatter,scatter src=explicit symbolic]table[meta=label] {
x y label
4 .000014 solveSmart
10 .000076 solveSmart
15 .000264 solveSmart
20 .001012 solveSmart
};


\addplot[scatter,scatter src=explicit symbolic]table[meta=label] {
x y label
4 .000012 solvePriceDynamic
10 .000548 solvePriceDynamic
15 .016784 solvePriceDynamic
20 .494500 solvePriceDynamic
};
\addplot[scatter,scatter src=explicit symbolic]table[meta=label] {
x y label
4 .000002 solveHungry
10 .000006 solveHungry
15 .000008 solveHungry
20 .000010 solveHungry
};
\addplot[scatter,scatter src=explicit symbolic]table[meta=label] {
x y label
4 .000002 solveSingle
10 .000006 solveSingle
15 .000008 solveSingle
20 .000010 solveSingle
};
\addplot[scatter,scatter src=explicit symbolic]table[meta=label] {
x y label
4 .000010 fptas
10 .000170 fptas
15 .000644 fptas
20 .002006 fptas
};
\addplot[scatter,scatter src=explicit symbolic]table[meta=label] {
x y label
4 .000022 solveStupid
10 .001562 solveStupid
15 .043078 solveStupid
20 1.507664 solveStupid
};
\addplot[scatter,scatter src=explicit symbolic]table[meta=label] {
x y label
4 .000016 solveSmart
10 .000092 solveSmart
15 .000268 solveSmart
20 .001024 solveSmart
};


\addplot[scatter,scatter src=explicit symbolic]table[meta=label] {
x y label
4 .000010 solvePriceDynamic
10 .000328 solvePriceDynamic
15 .001646 solvePriceDynamic
20 .004958 solvePriceDynamic
};
\addplot[scatter,scatter src=explicit symbolic]table[meta=label] {
x y label
4 .000002 solveHungry
10 .000004 solveHungry
15 .000006 solveHungry
20 .000010 solveHungry
};
\addplot[scatter,scatter src=explicit symbolic]table[meta=label] {
x y label
4 .000002 solveSingle
10 .000004 solveSingle
15 .000006 solveSingle
20 .000010 solveSingle
};
\addplot[scatter,scatter src=explicit symbolic]table[meta=label] {
x y label
4 .000010 fptas
10 .000140 fptas
15 .000546 fptas
20 .001994 fptas
};
\addplot[scatter,scatter src=explicit symbolic]table[meta=label] {
x y label
4 .000022 solveStupid
10 .001338 solveStupid
15 .041064 solveStupid
20 1.669862 solveStupid
};
\addplot[scatter,scatter src=explicit symbolic]table[meta=label] {
x y label
4 .000014 solveSmart
10 .000080 solveSmart
15 .000244 solveSmart
20 .000956 solveSmart
};


\addplot[scatter,scatter src=explicit symbolic]table[meta=label] {
x y label
4 3.911340 hungryStupidDeviation
10 2.209160 hungryStupidDeviation
15 1.676708 hungryStupidDeviation
20 .858728 hungryStupidDeviation
};
\addplot[scatter,scatter src=explicit symbolic]table[meta=label] {
x y label
4 31.760600 singleDeviation
10 2.209160 singleDeviation
15 1.676708 singleDeviation
20 .858728 singleDeviation
};
\addplot[scatter,scatter src=explicit symbolic]table[meta=label] {
x y label
4 6.724640 fptasDeviation
10 2.920400 fptasDeviation
15 2.143400 fptasDeviation
20 1.608108 fptasDeviation
};


\addplot[scatter,scatter src=explicit symbolic]table[meta=label] {
x y label
4 3.829840 hungryStupidDeviation
10 2.125580 hungryStupidDeviation
15 1.699950 hungryStupidDeviation
20 .905228 hungryStupidDeviation
};
\addplot[scatter,scatter src=explicit symbolic]table[meta=label] {
x y label
4 32.843600 singleDeviation
10 2.125580 singleDeviation
15 1.699950 singleDeviation
20 .905228 singleDeviation
};
\addplot[scatter,scatter src=explicit symbolic]table[meta=label] {
x y label
4 6.585160 fptasDeviation
10 2.922120 fptasDeviation
15 2.279220 fptasDeviation
20 1.654382 fptasDeviation
};



\section{Implementation}
All of these algorithms are implemented in Python 3.7, and all data were gathered on the Windows10
OS running on Intel(R) Core(TM) i7-6700HQ CPU @ 2.60GHz.


\subsection{Brute force}
While using the naive brute force algorithm, we will check all combinations
and evaluate if any of them matches the requirements. This is accomplished
by creating a virtual binary tree, for example with the use of recursion,
where we either insert an element or not. This approach will result in a
binary tree, whose leaves will contain all possible combinations of our items.


\subsection{Branch and bounds}
Branch and bounds method is optimised version of the brute force algorithm.
In this method, we are limiting the number of combinations we need to test
in two ways. The first optimisation we are using is stoping further attempts
once the total weight of the items used is greater than the capacity of our
knapsack.

In the second optimisation, we are comparing the value we are attempting
to achieve and the value which we can still achieve given our current
combination. To get the value which we can still achieve, we take the
value of our current combination and add the sum of the values of all
the remaining elements, which are yet to be considered for the current
combination.

The complexity of this algorithm is also $\mathbb{O}(2^n)$, as in the
worst case, such as when we get a list of elements with low weight and
low value with the last element of high weight and value, we are forced
to go through all of the possible combinations to ensure none of them is
viable for our requirements.


\subsection{Dynamic programing}
Our task was to use dynamic programming for solving the optimisation version of the 0/1 knapsack
problem, for which we have selected the decomposition by value, due to its similarity with the
FPTAS algorithm.

For more information about Dynamic programming or exact algorithm, feel free to look at the code
attached or \cite[Moodle textbook]{WEBSITE:dynamicKnapsack}.

\subsection{Greedy heuristics}
The greedy algorithm sorts the elements given at input by the value to weight ratio. It adds as many
elements as possible with the preference given to the items with the highest value and lowest weight.

This approach might not seem too precise, but it offers much lower complexity with $\mathbb{O}(n \log(n))$,
where the most complex task is to sort the data.

\subsection{Single most valuable item}
This is the fastest and the least precise algorithm of those we are considering in this paper.
The main idea behind this algorithm is to select one element of the highest value independent
on weight. This algorithm can be precise in cases, where we have multiple low-valued items and one of high value.

However, this algorithm is highly unreliable, and I wouldn't recommend using it in any case.
It might offer linear complexity, yet the average deviation appeared to be above 50\%
in most of our test cases and in one case even surpassed 125\%.

\subsection{FPTAS}

The FPTAS algorithm is basically the same as the decomposition by price used in dynamic programming.
The only difference here is that we lower the values of all the items, which lowers the complexity
of the algorithm and in this case, allows us to limit the imprecision. For the purposes of our measurement
we have set the highest deviation to be 0.5.

In order to gain precise data for selected maximal imprecision, we have used the following formula.

\begin{itemize}
    \item $C_M = $max$\{c_1,c_2, ... c_n\}$
    \item $K = \dfrac{\epsilon C_M}{n}$
    \item $c_i' = \lfloor\dfrac{c_i}{K}\rfloor$
\end{itemize}


\section{Data dependence}
\subsection{Datasets}
For datasets generation, we have used the provided generator\cite{WEBSITE:dataGen} with the following parameters.

Balanced dataset:\\
\texttt{\$GEN -N 500 -n \$i -W 38713 -C 38713 -m 0.8 -w bal -c uni}

Dataset with predominant light elements:\\
\texttt{\$GEN -N 500 -n \$i -W 38713 -C 38713 -m 0.8 -w light -c uni}

Dataset with predominant heavy elements:\\
\texttt{\$GEN -N 500 -n \$i -W 38713 -C 38713 -m 0.8 -w heavy -c uni}

Dataset with low correlation:\\
\texttt{\$GEN -N 500 -n \$i -W 38713 -C 38713 -m 0.8 -w bal -c corr}

Dataset with high correlation:\\
\texttt{\$GEN -N 500 -n \$i -W 38713 -C 38713 -m 0.8 -w bal -c strong}

Dataset with big knapsack:\\
\texttt{\$GEN -N 500 -n \$i -W 38713 -C 38713 -m 0.95 -w bal -c uni}

Dataset with small knapsack:\\
\texttt{\$GEN -N 500 -n \$i -W 38713 -C 38713 -m 0.3 -w bal -c uni}

Dataset with items of low weight and price:\\
\texttt{\$GEN -N 500 -n \$i -W 100 -C 100 -m 0.8 -w bal -c uni}

Dataset with items with low weight and price as well as small knapsack:\\
\texttt{\$GEN -N 500 -n \$i -W 100 -C 100 -m 0.3 -w bal -c uni}


\subsection{Dynamic algorithm}
As we can see in \cref{plot:balancedTime,plot:bigKnapsackTime,plot:correlationWeakTime,plot:heavyTime,plot:correlationStrongTime,plot:smallItemsTime,plot:smallKnapsackTime,plot:lightTime,plot:smallItemsSmallKnapsackTime},
the dynamic algorithm has fairly high time complexity. This relatively high complexity
is here due to our datasets having particularly high valued elements. Moreover, when we
set our maximal Weight and Price to lower values, we can see the complexity lowering significantly.

\subsection{FPTAS algorithm}
When we take a look at the fptas algorithm, we can see its indifference to any changes on our datasets.
This case might be caused by our implementation of the said algorithm, where we use a dictionary
for storing values instead of the usual array.

\subsection{Brute force algorithm}
The brute force algorithm is indifferent to any changes to a dataset by design, as it has to test
every option before returning our solution, thus having the same average time complexity in all
our test cases.

\subsection{Branch \& bounds algorithm}
As we observe the branch and bounds algorithm, we can see its relatively low complexity,
which is mostly thanks to our data having considerably high knapsack with the 0.8 ratio
between knapsacks load capacity and the combined weight of all the items. Thanks to this
setup, the B\&B algorithm can skip a significant part of combinations, as it can add most
of the elements. This difference can be well observed at
\cref{plot:bigKnapsackTime,plot:smallKnapsackTime}.

\subsection{Greedy heuristics}
The greedy heuristics, as well as its modification, shows noteworthy dependence on dataset
variations. When we take a look at \cref{plot:correlationWeakDeviation,plot:correlationStrongDeviation},
we can observe the growing imprecision, as items in our datasets correlate from our middle ground.

Moreover, we can see another divergence when we take a look at
\cref{plot:smallKnapsackDeviation,plot:bigKnapsackDeviation}. We can see noticeable imprecision,
which is caused by having a small knapsack, which in turn means that the algorithm has to select
a lesser amount of items from a bigger pool, which introduces the higher possibility of a mistake.

On the other hand, this dependency appears only while observing the imprecision. While we are
looking at the time complexity for each dataset, the greedy heuristics, as well as its modification,
shows minimal deviation, as it always holds on to its complexity of  $\Theta($log $n)$.


\section{Conclusion}
In conclusion, we have tested six different algorithms for solving the optimisation version
of the 0/1 knapsack problem on nine different datasets providing comprehensive examples for
various dependencies of imprecision, as well as time complexity to given alterations of our
datasets. Furthermore, we have described these dependencies, as well as provided multiple
examples with their graphic representation and substantiated these dependencies for eac
dataset and its variables.


\newpage
\bibliographystyle{iso690}
\nocite{*} % all entries in the bib file
\bibliography{database.bib}

\end{document}